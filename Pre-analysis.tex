\documentclass[11pt,a4paper,leqno]{article}
\usepackage{a4wide}
\usepackage[T1]{fontenc}
\usepackage[utf8]{inputenc}
\usepackage{float, afterpage, rotating, graphicx}
\usepackage{longtable, booktabs, tabularx}
\usepackage{verbatim}
\usepackage{eurosym, calc, chngcntr}
\usepackage{amsmath, amssymb, amsfonts, amsthm, bm, delarray} 
\usepackage{caption}
\usepackage{tkz-graph}
\usetikzlibrary{arrows,positioning,shapes,shapes.multipart,patterns,mindmap,shadows}

% \usepackage[backend=biber, natbib=true, bibencoding=inputenc, bibstyle=authoryear-ibid, citestyle=authoryear-comp, maxnames=10]{biblatex}
% \bibliography{bib/hmg}

\usepackage[unicode=true]{hyperref}
\hypersetup{colorlinks=true, linkcolor=black, anchorcolor=black, citecolor=black, filecolor=black, menucolor=black, runcolor=black, urlcolor=black}
\setlength{\parskip}{.5ex}
\setlength{\parindent}{0ex}

\theoremstyle{definition}
\newtheorem{exercise}{Exercise}
\renewcommand{\theenumi}{\roman{enumi}}

% Set this counter to "first exercise of the week minus one".
\setcounter{exercise}{0}

\begin{document}

\begin{center}
    \begin{large}
        \textbf{
        How can traumata influence the locus of control?\\
        Pre-analysis \\[2ex]}
    \end{large}
\end{center}

   ~ % do not delete this innocent tilde unless you put text here
   For investigating how traumata can influence the locus of control, we would conduct three specifications, a more general, a more specific one and one including just the number of traumata experienced. In all specifications we would use change in LOC as a dependent variable. In the GSOEP, LOC is measured by asking 10 questions and we would use a principal component analysis to reduce this number of items. The following controls will be included in all specifications: age, age\textsuperscript{2}, a female dummy, education dummies, ln(real income), a dummy for being religious, the time between the traumata happening and the following LOC measurement (to account for depreciation of the traumatic experience), year dummies and employment status. \\
   
   In the first more general specification of our analysis, we would include traumatic experiences concerning health, the social and the economic environment. We hypothesize that all of these experiences shift the locus of control towards a more external locus of control. Experiences concerning health are the birth of a child who is handicapped and the change in the degree of disability of a household member. Social events which could possibly influence the locus of control are a divorce, an unplanned pregnancy, a relationship breakup, the death of a partner, father, mother, child or a household member or any other family composition change. An experience concerning the economic environment could be the loss of a job which we would include as a dummy in our first specification.\\
   
   In our second, more precise specification, we would drop some of the variables from the three categories health, social and economic environment used before and include some in another way. Again, we hypothesize that they all lead to a more external locus of control. Concerning health, we would also include the birth of a handicapped child, but the change in the degree of disability of a household member would only be included in case of a change from 0 to 100 \%. The traumata concerning the social environment would also be an unplanned pregnancy, but neither a divorce, nor a relationship break-up as we suppose that the latter incidents are in most cases not exogenous and would therefore not necessarily lead to a more external locus of control. Moreover, regarding deaths, we would drop deaths where the person investigated is not strongly affected by the loss. In terms of the economic environment, we would include only the job losses which are rather exogenous e.g. when the place of work closed or the person was dismissed by his or her employer, but not if there was e.g. a mutual agreement with the employer or the job was ended due to retirement.    \\
   
   In order to find out whether the relationship between change in locus of control and negative life events varies with the number of negative life events reported, we include the total number of traumata in our third specification instead of using all the dummies for the different traumatic experiences. On top of that, we also include the total number squared to test for any non-linearities. In this specification, we do not control for the time between the occurences of the traumatas and the LOC measurement. \\
   
   As LOC is measured every five years in the GSOEP from 2005 onwards (in the youth questionnaire every year from 2006 onwards), we would need the GSOEP data from 2005-2016 for our analysis. The variables we need are part of the individual questionnaire, the deceased persons questionnaire, mother-child questionnaire (0-3 years) and the youth questionnaire. 
   
   \begin{itemize}
   	\item possible other controls: married? children? first measured level of LOC (Buddelmeyer, Powdthavee (2016): people with a more internal LOC suffer less from emotional pain after a negative shock - but: does this mean that their LOC is also less affected??)? do we really want to control for income? 
   	\item Do we want to write the equations out? 
   	\item Which SEs do we use?
   	\item How do we handle missings?
   	\item Ich glaube, die letzten Punkte erwartet er nicht, aber falls du zu irgendwas schon eine Idee hast, kannst du es ja einfügen. 
   \end{itemize}


% Example for inheritance diagram from the lecture.
%
% \begin{tiny}
%     \begin{tikzpicture}
%         \node (1) [
%             rectangle split,
%             rectangle split parts=6,
%             draw,
%             text width=6.00cm,
%             shift={(-1.25,2.0)}
%         ]
%         {
%             \nodepart{one}
%             \begin{small}
%             \textbf{AgentRiskyProspectsWithGambles}
%             \end{small}
%             \nodepart{two}
%             certainty\_equivalent\_gambles()
%             \nodepart{three}
%             gambles \textcolor{red}{[tuple]}
%             \nodepart{four}
%             certainty\_equivalent(prospects, probabilities)
%             \nodepart{five}
%             expected\_utility(prospects, probabilities)
%             \nodepart{six}
%             \_\_init\_\_(gambles)
%         };
%         \node (2) [
%             rectangle split,
%             rectangle split parts=5,
%             draw,
%             text width=4.5cm,
%             shift={(-3,-2.0)}
%         ]
%         {
%             \nodepart{one}
%             \begin{small}
%             \textbf{AgentKinkedWithGambles}
%             \end{small}
%             \nodepart{two}
%             utility(z)
%             \nodepart{three}
%             utility\_inverse(x)
%             \nodepart{four}
%             lambda\_ \textcolor{red}{[float]}
%             \nodepart{five}
%             \_\_init\_\_(lambda\_, gambles)
%         };
%         \node (3) [
%             rectangle split,
%             rectangle split parts=5,
%             draw,
%             text width=3.5cm,
%             shift={(4.25,1.0)}
%         ]
%         {
%             \nodepart{one}
%             \begin{small}
%             \textbf{Gamble}
%             \end{small}
%             \nodepart{two}
%             prospects()
%             \nodepart{three}
%             probabilities()
%             \nodepart{four}
%             name \textcolor{red}{[str]}
%             \nodepart{five}
%             \_\_init\_\_(gamble\_dict, name)
%         };
%         \draw[->] (2) to [out=150, in=162] (1);
%         \draw[->] (1) to [in=140, out=2] (3);
%     \end{tikzpicture}
% \end{tiny}

% \printbibliography 

\end{document}