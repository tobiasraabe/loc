\documentclass[12pt,a4paper,fleqn]{article}
\usepackage[english]{babel}
%\usepackage[utf8]{inputenc}
\usepackage{ae} % falls in acrobat reader nicht schön dargestellt werden kann
\usepackage{amsmath} % für mathematische symbole
\usepackage{amsfonts}
\usepackage{amssymb}
\usepackage{graphicx}
\usepackage{tabularx}
\usepackage[left=3.0cm,top=2.0cm,right=2.0cm,bottom=2.0cm]{geometry}
\usepackage{filecontents}
\usepackage{setspace}
\usepackage{booktabs}
\usepackage{multicol}
\usepackage{appendix}
\usepackage{subfigure}
\usepackage{supertabular}
\usepackage{caption} 
\usepackage{array, multirow}
\usepackage{lscape}
\usepackage{fixltx2e}
\usepackage[scaled]{uarial}
\usepackage{placeins} %Fuer den FloatBarrier Befehl
\usepackage[latin1]{inputenc}
\usepackage{diagbox}
\renewcommand{\baselinestretch}{1.44} % entspricht 1,5-fachem Zeilenabstand in Microsoft Word
\usepackage{natbib} % Usepackage fuer Bibliographie
\bibliographystyle{apalike} % Usepackage fuer Bibliographie
\renewcommand{\footnotesize}{\scriptsize}

\pagenumbering{roman} 
\begin{document}
	\begin{titlepage}
		\begin{center}
			\vspace*{35mm}
			{\bfseries\Large{Malleability of Locus of Control by Traumatic Events}\\
			\vspace{25mm}}
			\begin{spacing}{1.3}
				\large Research Module in Management and Applied Microeconomics\\
				\vspace{65mm}
			   Prof. Dr. Thomas Dohmen, Dr. Philipp Eisenhauer \\
				\vspace{10mm}
				January 2018   \\
				\textbf{Tobias Raabe, Imke Strampe} \\
			\end{spacing}
		\end{center}
	\end{titlepage}
\tableofcontents

\pagebreak

\listoffigures  \addcontentsline{}{Abbildungsverzeichnis} \clearpage 
\listoftables  \addcontentsline{}{Tabellenverzeichnisverzeichnis} \clearpage 

\pagebreak

\pagenumbering{arabic}

\section{Introduction}
Character traits, cognitive and non-cognitive abilities have been increasingly embedded in economic models during the last one or two decades. In this paper we focus on one specific personality trait, the locus of control, which goes back to the work of Rotter (1966). The locus of control refers to the causality between an individual's behaviour and its consequences. Individuals with a rather internal locus of control think that they can entirely influence the outcome of their behaviour and actions whereas individuals with a more external locus of control are of the opinion that fate or luck mainly determine what happens to them in life.
The locus of control is especially relevant in economic applications as it is a major determinant of a wide range of life outcomes. \\
According to Heineck and Anger (2010), having a more internal locus of control is associated with higher wages. Piatek and Pinger (2010) discover that education is the main channel for this phenomenon as individuals with an internal locus of control invest more in education. Gallo et al. (2013) find that unemployment durations following a job loss are also lower for these people. Besides, Cobb-Clark et al. (2014) observe that individuals with an internal locus of control are more likely to eat healthier and exercise regularly. Another benefit of having an internal locus of control is found by Buddelmeyer, Powdthavee (2016) who discover that an internal locus of control increases the individual's psychological resilience against certain negative shocks. 
In the literature, adults' locus of control is often regarded as being stable or even fixed over time (see for example Heineck, Anger (2010), Semykina, Linz (2007)). By treating it as exogenous, reverse causality problems are overcome as it is assumed not to be influenced by any labour market outcomes. This assumption is convenient if the locus of control is measured only once in the data, but the results are likely to be biased if the locus of control is indeed not stable over the relevant time frame. \\
In this study, we want to challenge this assumption by investigating whether an individual's locus of control changes after he or she has experienced an exogenously caused traumatic event (e.g. the death of a family member). \\
This study is in line with the early work of Alesina, Ferrara (2002) who look at changes in trust due to traumatic events. An extension of the framework to locus of control was done by Cobb-Clark, Schurer (2013) using Australian data (HILDA survey) who find that positive and negative life events have only a minor and empirically irrelevant  influence on the locus of control. 
Preuss, Hennecke (2017) give a brief overview of the studies focusing on the relationship between unemployment and locus of control. Using the GSOEP, Preuss, Hennecke (2017) provide evidence with greater external validity. By differentiating between stated locus of control, which is context-specific, and actual locus of control, they do not reject the stability assumption of actual locus of control. This is due to the fact that they observe a change in the stated locus of control during unemployment which vanishes after reemployment. \\
Instead of focusing only on unemployment, we look at traumatas concerning three categories of life - the health, the social and the economic environment.  For this purpose, we make use of the German Socio-Economic Panel (SOEP, 2016) and run three different regression specifications. We expect all of the traumatic events to lead to a shift towards a more external locus of control.\\
The outline of the paper is as follows. Section 2 describes the data and analyses possible causal channels of locus of control. Section 3 explains the estimation strategy and section 4 presents the result. We conclude with section 6. 



\appendix
\setcounter{secnumdepth}{0}
\section{Appendix}



\pagebreak
\setcounter{secnumdepth}{0}
\section{References}


	
	
\pagebreak
\setcounter{secnumdepth}{0}
\section{Declaration}




\end{document}
