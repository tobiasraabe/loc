
\documentclass[12pt, a4paper, fleqn, parskip]{scrartcl}
\def\datum{\today}
\author{Tobias Raabe and Imke Strampe}
\title{Malleability of Locus of Control by Traumatic Events}

\usepackage[english]{babel}  % set language to adjust headings, automatic hypenation
\usepackage[utf8]{inputenc}  % allows to use umlauts and certain symbols
\usepackage[T1]{fontenc}  % typesetting, necessary for umlauts and accentuations
\usepackage{lmodern}  % Latin Modern Fonts are successors of standard fonts
\usepackage{amsmath, amssymb, amsthm}
% Packages for tables
\usepackage{booktabs}
\usepackage{caption}
\usepackage{array}
\newcommand{\rowgroup}[1]{\hspace{-1em}#1}  % Allows for unindented groups in tables
\usepackage{float}

\usepackage{graphicx}
\usepackage[space]{grffile}
\usepackage{hyperref}
\usepackage{setspace}
\usepackage{xcolor}
\hypersetup{
    colorlinks,
    linkcolor={red!50!black},
    citecolor={blue!50!black},
    urlcolor={blue!80!black}
}

\usepackage[backend=bibtex, natbib=true, bibencoding=utf8, safeinputenc=true,
bibstyle=authoryear, citestyle=authoryear-comp, maxcitenames=2, mincitenames=1,
uniquelist=false, useprefix=true, maxbibnames=99, minbibnames=99, backref=true,
backrefstyle=three, doi=true, isbn=false, sortcites=false, dashed=false,
firstinits=true]{biblatex}

\addbibresource{references.bib}

\pagenumbering{roman}
\begin{document}
	\begin{titlepage}
		\begin{center}
			\vspace*{35mm}
			{\bfseries\Large{Malleability of Locus of Control by Traumatic Events}\\
			\vspace{25mm}}
			\begin{spacing}{1.3}
				\large Research Module in Management and Applied Microeconomics\\
				\vspace{65mm}
			   Prof. Dr. Thomas Dohmen, Dr. Philipp Eisenhauer \\
				\vspace{10mm}
				January 2018   \\
				\textbf{Tobias Raabe, Imke Strampe} \\
			\end{spacing}
		\end{center}
	\end{titlepage}
\tableofcontents

\pagebreak

\listoffigures  \addcontentsline{}{List of Figures} \clearpage
\listoftables  \addcontentsline{}{List of Tables} \clearpage

\pagebreak

\pagenumbering{arabic}

\section{Introduction}

Character traits, cognitive and non-cognitive abilities have been increasingly
embedded in economic models during the last one or two decades. In this paper
we focus on one specific personality trait, the locus of control, which goes
back to the work of \citet{rotter1966}. The locus of control refers to the
causality between an individual's behavior and its consequences. Individuals
with a rather internal locus of control think that they can entirely influence
the outcome of their behavior and actions whereas individuals with a more
external locus of control are of the opinion that fate or luck mainly determine
what happens to them in life. The locus of control is especially relevant in
economic applications as it is a major determinant of a wide range of life
outcomes.

According to \citet{heineck2010}, having a more internal locus of control is
associated with higher wages. \citet{piatek2010} discover that education is the
main channel for this phenomenon as individuals with an internal locus of
control invest more in education. Gallo et al. (2013) find that unemployment
durations following a job loss are also lower for these people. Besides, Cobb-
Clark et al. (2014) observe that individuals with an internal locus of control
are more likely to eat healthier and exercise regularly. Another benefit of
having an internal locus of control is found by \citet{buddelmeyer2016} who
discover that an internal locus of control increases the individual's
psychological resilience against certain negative shocks. In the literature,
adults' locus of control is often regarded as being stable or even fixed over
time \citep{heineck2010,semykina2007}. By treating it as exogenous, reverse
causality problems are overcome as it is assumed not to be influenced by any
labour market outcomes. This assumption is convenient if the locus of control
is measured only once in the data, but the results are likely to be biased if
the locus of control is indeed not stable over the relevant time frame. For an
elaborate explanation of the econometric issues that can arise, see
\citet{cobb2013}.

In this study, we want to challenge this assumption by investigating whether an
individual's locus of control changes after he or she has experienced an
exogenously caused traumatic event (e.g. the death of a family member).

This study is in line with the early work of \citet{alesina2002} who look at
changes in trust due to traumatic events. An extension of the framework to
locus of control was done by \citet{cobb2013} using Australian data
(HILDA survey) who find that positive and negative life events have only a
minor and empirically irrelevant  influence on the locus of control.
\citet{preuss2017} give a brief overview of the studies focusing on the
relationship between unemployment and locus of control. Using the GSOEP,
\citet{preuss2017} provide evidence with greater external validity. By
differentiating between stated locus of control, which is context-specific, and
actual locus of control, they do not reject the stability assumption of actual
locus of control. This is due to the fact that they observe a change in the
stated locus of control during unemployment which vanishes after reemployment.

Instead of focusing only on unemployment, we look at traumata concerning three
categories of life - the health, the social and the economic environment.  For
this purpose, we make use of the German Socio-Economic Panel (SOEP, 2016) and
run three different regression specifications. By using a differences-in-
differences approach, we expect all of the traumatic events to lead to a shift
towards a more external locus of control.

The outline of the paper is as follows. Section 2 describes the data and
analyses possible causal channels of locus of control. Section 3 explains the
estimation strategy and section 4 presents the result. We conclude with section
6.

\section{Theoretical Considerations} % (fold)
\label{sec:theoretical_considerations}

\textcolor{red}{The following paragraph is pretty close to PH 2017.}

Previous literature has raised the importance of measurement error when one is
trying to access a personality trait. \citet{borghans2008} raise awareness that
measurements are only imperfect proxies of real personality traits. This
problem leads to an attenuation bias if the locus of control is used as the
independent variable. In our example where locus of control is the dependent
variable and if we assume that the error in measuring locus of control is
unrelated to everything else, standard errors increase and may cover
significant effects. \citet{rammstedt2010} and \citet{rammstedt2011} discuss
cases where the measurement error might not be independent of other variables.
They highlight the case that compliance to respond to Big Five items is more
prevalent among less well educated subjects. \citet{golsteyn2017} investigate
potential anchoring effects in personality interviews.

% section theoretical_considerations (end)

\section{Data}

Our data comes from the German Socio-Economic Panel (GSOEP)  \citep{soep2017}
which is an annual representative household panel study. It focuses on all
participants from 2005-2015 which is about 16,000 households or 30,000 people.

\subsection{Locus of Control} % (fold)
\label{sub:locus_of_control}

Our main sample restriction is the elicitation of the locus of control survey
questions which are asked 2005, 2010 and 2015. In these waves the
survey participants have been asked to what extent they agree or disagree with
the statements presented in Table~\ref{tab:loc_items} (1 = disagree completely;
7 = agree completely).

\begin{table}[h!]
	\begin{tabular}{p{1.5cm} p{14cm}}
	\toprule
	Question: & \textit{The following statements apply to different attitudes
	towards life and the future. To what degree to you personally agree with
	the following statements? Please answer according to the following scale: 1
	means disagree completely, and 7 means agree completely.} \\
	\midrule
	1  & How my life goes depends on myself \\
	2  & Compared to others, I haven't achieved what I deserved \\
	3  & What one achieves in life is mainly a question of luck or fate \\
	4  & If a person is socially or politically active, he/she can have an effect on social conditions \\
	5  & I often have the experience that others make decisions regarding my life \\
	6  & One has to work hard in order to succeed \\
	7  & When I encounter difficulties, I have doubts about my abilities \\
	8  & The opportunities I have in life are determined by social conditions \\
	9  & Inborn abilities are more important than any efforts one can make \\
	10 & I have little control over the things that happen in my life \\
	\bottomrule
	\end{tabular}
	\caption{Source: SOEP 2005, 2010, 2015}
	\label{tab:loc_items}
\end{table}

Following the specification of \citet{specht2013}, we only focus on items 1, 2,
3, 5, 7, 8, 10. We also recoded all items so that higher numbers reflect a
greater feeling of control. A calculation of Cronbach's Alpha, a metric for the
internal consistency of a scale, yields a value of 0.698 for the seven item
scale. This is an improvement of 0.091 compared to the same metric applied to
the ten item scale. The measure can yield values in $]-\infty, 1]$ where values
higher than 0.65 are acceptable as a rule of thumb. \textcolor{red}{(see
Wieland 2017 statistical and judgemental, and Podsakoff 2016 recommentdations
for more information.)}

As \citet{piatek2016} find that using an index to construct a measure of
locus of control leads to measurement error and attenuation bias, we conduct a
... analysis for getting a more valid measure.

\begin{figure}[ht!]
	\centering
	\includegraphics[width=\textwidth]{../../out/figures/fig-fa-ten-comp-explained-variance}
	\caption{Principal Component Analysis - Explained Variance}
	\label{fig:pca_ev}
\end{figure}

% subsection locus_of_control (end)


\subsection{Events} % (fold)
\label{sub:events}

To explore the malleability of locus of control, we identify several events
which might push the individual towards a more external locus of control. The
reasoning is that an exogenous shock increases the salience of lack of power
which surely affects the context specific part

% subsection events (end)

For the traumatic experiences concerning health, we include a dummy which is
one if someone has born a child who has confirmed disorders and which is zero
otherwise. On top of that, we use the change in disability of a household
member in one specification and a dummy which is one only if the change is from
one to hundred percent in another specification.

Traumata relating to the social environment are a divorce dummy, a relationship
breakup dummy, an unplanned pregnancy dummy, a dummy for the death of a
partner, mother, father, child or a household member.

We also make use of a traumatic experience concerning the economic environment
which is a dummy for the loss of a job.

\textit{sample restrictions, missing values, how many people suffered from how
many traumata}

We expect to see a shift towards a more external locus of control after the
occurrence of each of the traumata. If such an exogenously caused, dreadful
incident happens to someone, the person might wonder why he or she was
affected. He might feel as a victim, become more doubtful and have the
impression that he does not deserve that. Besides, his feeling of having
control over what happens to him would shrink. Consequently he would start
believing more in fate than in his own ability of mastering everything and the
locus of control would shift towards externality. The entire process could
happen subconsciously.

\subsection{Controls} % (fold)
\label{sub:controls}



% subsection controls (end)

\section{Estimation Strategy}

For investigating the impact of traumata on the locus of control, we conduct
three specifications, a more general, a more specific one and one including
just the number of traumata experienced. In all specifications we use change in
locus of control as a dependent variable and estimate the effect by a
differences-in-differences analysis. Regarding the controls, we follow the
specifications by \citet{cobb2013}: dummies for age groups, marital
status, immigrant status, employment status, educational qualification dummies
and log household income. Furthermore, we include a gender dummy and dummies
for years into the regression.

We would also like to control for the time passed since the last shock to
identify depreciation effects of traumatic events.

In the first more general specification of our analysis, we include the birth
of a child who is handicapped and the change in the degree of disability of a
household member. On top of that, we use the occurrence of a divorce, an
unplanned pregnancy, a relationship breakup, the death of a partner, father,
mother, child or a household member or any other family composition change.
Besides, the loss of a job would be included in our first specification.

In our second, more precise specification, we include the change in the degree
of disability of a household member only in case of a change from 0 to 100\%.
Moreover, we drop the occurrence of a divorce and of a relationship break-up as
we suppose that the latter incidents are in most cases not exogenous and would
therefore not necessarily lead to a more external locus of control. Regarding
deaths, we would drop deaths where the person investigated is not strongly
affected by the loss. On top of that, we include only the job losses which are
rather exogenous e.g. when the place of work closed or the person was dismissed
by his or her employer, but not if there was e.g. a mutual agreement with the
employer or the job was ended due to retirement.

In order to find out whether the relationship between change in locus of
control and negative life events varies with the number of negative life events
reported, we include the total number of traumas in our third specification
instead of using all the dummies for the different traumatic experiences.
Additionally, we include the total number squared to test for any non-
linearities. In this specification, we do not control for the time between the
occurrences of the traumata and the locus of control measurement.

In our first and second specification, we estimate the following equation

\begin{align}
	\Delta LOC \textsubscript{i,t}= \alpha + X \textsubscript{i,t} * \beta + Z
	\textsubscript{i,t} * \gamma + Y\textsubscript{t} * \delta +
	\epsilon\textsubscript{i}
\end{align}

where $\alpha$ is the average change in locus of control, $
X\textsubscript{i,t}$ a vector of the traumata dummies, $ Z\textsubscript{i,t}$
a vector in which the control variables are included, $ Y\textsubscript{t}$
comprises the year dummies and $\epsilon\textsubscript{i}$ is the error term.

The vector $\beta$ measures the average treatment effect on the treated which
shows how each traumatic experience changes the change in locus of control over
time on average. In our first specification, in which the traumata are not
necessarily exogenous, $\beta $ is clearly the effect only on the treated as
there could be some possible selection. In our second specification, we make
sure that all traumata are exogenously caused and therefore there should be no
selection in or out of treatment. Consequently, we would expect $\beta$ to
measure the average treatment effect without any conditioning.

In our third specification, the estimated equation is

\begin{align}
	\Delta LOC \textsubscript{i,t}= \alpha + N \textsubscript{i,t} * \beta + N
	\textsuperscript{2}\textsubscript{i,t} * \eta + Z \textsubscript{i,t} *
	\gamma + Y\textsubscript{t} * \delta + \epsilon\textsubscript{i}
\end{align}

in which we include the number of traumatic experiences $N \textsubscript{i,t}$
and the number of traumatic experiences squared $N
\textsuperscript{2}\textsubscript{i,t}$.

% We use clustered standard errors on the individual level to account for
% correlation between the same individual over time in all three
% specifications.

We use clustered standard errors on the household level to account for the fact
that traumatic experiences affect the entire household in most cases. As the
same individual is correlated over time, we would like to include individual
fixed effects. Unfortunately, this is not possible as we observe some
individuals only for one period. It would be interesting to see whether
reducing the sample to only those individuals which we observe for two periods
and including individual fixed effects for them would change our results.


\section{Results}

\begin{table}
	\tiny{\begin{center}
 \begin{tabular}{lclc}
 \toprule
 \textbf{Dep. Variable:}                                                                                              & Change in Locus of Control & \textbf{  R-squared:         } &     0.005   \\
 \textbf{Model:}                                                                                                      &        OLS         & \textbf{  Adj. R-squared:    } &     0.003   \\
 \textbf{Method:}                                                                                                     &   Least Squares    & \textbf{  F-statistic:       } &     1.846   \\
 \textbf{Date:}                                                                                                       &  Thu, 25 Jan 2018  & \textbf{  Prob (F-statistic):} &  3.71e-05   \\
 \textbf{Time:}                                                                                                       &      13:10:54      & \textbf{  Log-Likelihood:    } &   -26849.   \\
 \textbf{No. Observations:}                                                                                           &        21371       & \textbf{  AIC:               } & 5.380e+04   \\
 \textbf{Df Residuals:}                                                                                               &        21320       & \textbf{  BIC:               } & 5.421e+04   \\
 \textbf{Df Model:}                                                                                                   &           50       & \textbf{                     } &             \\
 \bottomrule
 \end{tabular}
 \begin{tabular}{lcccccc}
                                                                                                                      & \textbf{coef} & \textbf{std err} & \textbf{t} & \textbf{P$>$$|$t$|$} & \textbf{[0.025} & \textbf{0.975]}  \\
 \midrule
 \textbf{Intercept}                                                                                                   &      -0.0319  &        0.054     &    -0.595  &         0.552        &       -0.137    &        0.073     \\
 \textbf{Child Has Disorders}                                                                           &       0.0988  &        0.113     &     0.877  &         0.381        &       -0.122    &        0.320     \\
 \textbf{Death of Child}                                                                              &      -0.1382  &        0.091     &    -1.513  &         0.130        &       -0.317    &        0.041     \\
 \textbf{Death of Father}                                                                             &       0.0211  &        0.038     &     0.549  &         0.583        &       -0.054    &        0.096     \\
 \textbf{Death of HH Person}                                                                         &      -0.0449  &        0.094     &    -0.476  &         0.634        &       -0.230    &        0.140     \\
 \textbf{Death of Mother}                                                                             &      -0.0216  &        0.039     &    -0.555  &         0.579        &       -0.098    &        0.055     \\
 \textbf{Death of Partner}                                                                            &      -0.0493  &        0.054     &    -0.913  &         0.361        &       -0.155    &        0.056     \\
 \textbf{Divorce}                                                                                  &      -0.2960  &        0.276     &    -1.072  &         0.284        &       -0.837    &        0.245     \\
 \textbf{HH Composition Changed}                                                                          &      -0.0285  &        0.056     &    -0.505  &         0.614        &       -0.139    &        0.082     \\
 \textbf{Displacement}                                                                          &      -0.0222  &        0.032     &    -0.693  &         0.488        &       -0.085    &        0.041     \\
 \textbf{Legally Handicapped}                                                                &      -0.1187  &        0.093     &    -1.280  &         0.200        &       -0.301    &        0.063     \\
 \textbf{Unplanned Pregnancy}                                                                      &      -0.0525  &        0.065     &    -0.808  &         0.419        &       -0.180    &        0.075     \\
 \textbf{Separation}                                                                                 &       0.1072  &        0.177     &     0.607  &         0.544        &       -0.239    &        0.453     \\
 \textbf{Divorce * Single}                                                     &       0.3002  &        0.326     &     0.920  &         0.358        &       -0.340    &        0.940     \\
 \textbf{Displacement * Not Employed}                                    &      -0.0142  &        0.031     &    -0.454  &         0.650        &       -0.076    &        0.047     \\
 \textbf{Displacement * Other Empl. Status}                                           &      -0.0258  &        0.058     &    -0.444  &         0.657        &       -0.140    &        0.088     \\
 \textbf{Separation * Single}                                                    &      -0.0988  &        0.212     &    -0.465  &         0.642        &       -0.515    &        0.318     \\
 \textbf{Child Disorders * Time Since Occurrence}                                          &      -0.0060  &        0.004     &    -1.489  &         0.136        &       -0.014    &        0.002     \\
 \textbf{Child Death * Time Since Occurrence}                                                &      -0.0010  &        0.006     &    -0.149  &         0.882        &       -0.013    &        0.012     \\
 \textbf{Father Death * Time Since Occurrence}                                              &      -0.0025  &        0.001     &    -1.705  &         0.088        &       -0.005    &        0.000     \\
 \textbf{Household Person Death * Time Since Occurrence}                                      &       0.0067  &        0.005     &     1.236  &         0.217        &       -0.004    &        0.017     \\
 \textbf{Mother Death * Time Since Occurrence}                                              &       0.0021  &        0.002     &     1.342  &         0.179        &       -0.001    &        0.005     \\
 \textbf{Partner Death * Time Since Occurrence}                                            &       0.0043  &        0.003     &     1.586  &         0.113        &       -0.001    &        0.010     \\
 \textbf{Divorce * Time Since Occurrence}                                                        &       0.0126  &        0.011     &     1.154  &         0.248        &       -0.009    &        0.034     \\
 \textbf{Divorce * Single * Time Since Occurrence}                           &      -0.0087  &        0.011     &    -0.775  &         0.438        &       -0.031    &        0.013     \\
 \textbf{HH Composition Changed * Time Since Occurrence}                                        &      -0.0011  &        0.003     &    -0.381  &         0.703        &       -0.007    &        0.005     \\
 \textcolor{red}{\textbf{Displacement * Time Since Occurrence}}                                        &       \textcolor{red}{0.0024}  &        \textcolor{red}{0.001}     &     \textcolor{red}{2.547}  &         \textcolor{red}{0.011}        &        \textcolor{red}{0.001}    &        \textcolor{red}{0.004}     \\
 \textbf{Displacement * Time Since Occurrence * Not Employed}  &      -0.0018  &        0.001     &    -1.270  &         0.204        &       -0.005    &        0.001     \\
 \textbf{Displacement * Time Since Occurrence * Other Empl. Status}         &      -0.0005  &        0.003     &    -0.190  &         0.849        &       -0.006    &        0.005     \\
 \textbf{Handicapped * Time Since Occurrence}                    &       0.0025  &        0.005     &     0.528  &         0.597        &       -0.007    &        0.012     \\
 \textbf{Unplanned Pregnancy * Time Since Occurrence}                                &       0.0014  &        0.003     &     0.440  &         0.660        &       -0.005    &        0.008     \\
 \textbf{Separation * Time Since Occurrence}                                                      &      -0.0029  &        0.008     &    -0.375  &         0.708        &       -0.018    &        0.012     \\
 \textbf{Separation * Time Since Occurrence * Single}                         &       0.0036  &        0.008     &     0.456  &         0.649        &       -0.012    &        0.019     \\
 \end{tabular}
 %\caption{OLS Regression Results}
 \end{center}}
	\caption{First Specification}
	\label{tab:first_reg}
\end{table}
\newpage


\begin{table}
	\tiny{\input{../../out/tables/reg_table_2.tex}}
	\caption{Second Specification}
	\label{tab:second_reg}
\end{table}
\newpage

\begin{table}
	\input{../../out/tables/reg_table_3.tex}
	\caption{Third Specification}
	\label{tab:third_reg}
\end{table}
\newpage

\section{Results} % (fold)
\label{sec:results}

\subsection{Internal vs. External types} % (fold)
\label{sub:internal_vs_external_types}

% subsection internal_vs_external_types (end)

\subsection{Time Effects} % (fold)
\label{sub:time_effects}

In this section we want to improve on our results regarding events and time
effects. In the former regressions we included time variables which represent
the time in month since the last event to control for differences in time
between events and locus of control elicitations. The channel of time regarding
traumatic events is ambiguous as there are two possible fringe cases. First,
the salience of the traumatic event fades as time passes and the effect on
locus of control might wash out. For example, the death of a close relative
becomes more irrelevant as an individual proceeds with his daily life and
overcomes the sadness. Second, salience may exacerbate with more time as the
conflict is still unresolved. When a person looses her job and does not find
immediate reemployment, the doubts about her own ability might increase.

Our former estimation sample might not be sufficiently specified to assess this
problem. As we have restricted our sample to the period 2005-2015, we have not
captured information about events before 2005 or 2010 if a person enters the
sample in 2005 or 2010, respectively. To overcome this problem with our data,
we analyze only individuals which are observed from 2005 to 2015. This group
comprises NUMBER\_OF\_OBS observations. Next, we restrict the group to all
observations who experienced no traumatic event in the first period of
2005-2010. Our final sample contains NUM\_OBS observations. At last, we modify
the third regression specification by splitting the continuous time variable
for each event into five dummies where each of the dummies represents a twelve
month interval.

Our assumptions are that lagged effects of events do not last longer than five
years and that the sample bias due to the restrictions is not driven by
important factors.

The following graph shows the time interval dummies on the x-axis and the score
of the coefficients on the y-axis. Each of the lines represents the score of
the event as time increases.

INCLUDE GRAPHIC

More stuff.

% subsection time_effects (end)

% section results (end)

\section{Conclusion} % (fold)
\label{sec:conclusion}

\begin{itemize}
	\item improvements
	\begin{itemize}
		\item bigger sample size with 1999 and youth questionnaires
		\item make investigation of time effects cleaner by restricting sample
		to observations without event in the previous 5 years, etc.
	\end{itemize}
\end{itemize}
% section conclusion (end)

\printbibliography

\appendix
\setcounter{secnumdepth}{0}
\section{Appendix}

\subsection{Descriptive Statistics} % (fold)
\label{sub:descriptive_statistics}

\input{../../out/tables/tab-panel-descriptive-statistics-any}
\input{../../out/tables/tab-panel-descriptive-statistics-child-disorder}
\input{../../out/tables/tab-panel-descriptive-statistics-death-child}
\input{../../out/tables/tab-panel-descriptive-statistics-death-father}
\input{../../out/tables/tab-panel-descriptive-statistics-death-hh-person}
\input{../../out/tables/tab-panel-descriptive-statistics-death-mother}
\input{../../out/tables/tab-panel-descriptive-statistics-death-partner}
\input{../../out/tables/tab-panel-descriptive-statistics-divorced}
\input{../../out/tables/tab-panel-descriptive-statistics-hh-comp-change}
\input{../../out/tables/tab-panel-descriptive-statistics-last-job-ended}
\input{../../out/tables/tab-panel-descriptive-statistics-last-job-ended-limited}
\input{../../out/tables/tab-panel-descriptive-statistics-legally-handicapped-perc}
\input{../../out/tables/tab-panel-descriptive-statistics-pregnancy-unplanned}
\input{../../out/tables/tab-panel-descriptive-statistics-separated}

% subsection descriptive_statistics (end)

\newpage
\setcounter{secnumdepth}{0}
\section{Declaration}

\end{document}
