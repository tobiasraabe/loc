
\documentclass[12pt, a4paper, fleqn, parskip]{scrartcl}
\def\datum{\today}
\author{Tobias Raabe and Imke Strampe}
\title{Malleability of Locus of Control by Traumatic Events}

\usepackage[english]{babel}  % set language to adjust headings, automatic hypenation
\usepackage[utf8]{inputenc}  % allows to use umlauts and certain symbols
\usepackage[T1]{fontenc}  % typesetting, necessary for umlauts and accentuations
\usepackage{lmodern}  % Latin Modern Fonts are successors of standard fonts
\usepackage{amsmath, amssymb, amsthm}
% Packages for tables
\usepackage{booktabs}
\usepackage{caption}
\usepackage{array}
\newcommand{\rowgroup}[1]{\hspace{-1em}#1}  % Allows for unindented groups in tables
\usepackage{float}

\usepackage{graphicx}
\usepackage[space]{grffile}
\usepackage{hyperref}
\usepackage{setspace}
\usepackage{xcolor}
\hypersetup{
    colorlinks,
    linkcolor={red!50!black},
    citecolor={blue!50!black},
    urlcolor={blue!80!black}
}

\usepackage[backend=biber, natbib=true, bibencoding=utf8, safeinputenc=true,
bibstyle=authoryear, citestyle=authoryear-comp, maxcitenames=2, mincitenames=1,
uniquelist=false, useprefix=true, maxbibnames=99, minbibnames=99, backref=true,
backrefstyle=three, doi=true, isbn=false, sortcites=false, dashed=false,
giveninits=true, uniquename=init]{biblatex}

\addbibresource{references.bib}

\pagenumbering{roman}
\begin{document}
    \begin{titlepage}
        \begin{center}
            \vspace*{35mm}
            {\bfseries\Large{Malleability of Locus of Control by Traumatic Events}\\
            \vspace{25mm}}
            \begin{spacing}{1.3}
                \large Research Module in Management and Applied Microeconomics\\
                \vspace{65mm}
               Prof. Dr. Thomas Dohmen, Dr. Philipp Eisenhauer \\
                \vspace{10mm}
                January 2018   \\
                \textbf{Tobias Raabe, Imke Strampe} \\
            \end{spacing}
        \end{center}
    \end{titlepage}

\tableofcontents

\pagebreak

\listoffigures  \addcontentsline{}{List of Figures} \clearpage
\listoftables  \addcontentsline{}{List of Tables} \clearpage

\pagebreak

\pagenumbering{arabic}

\section{Introduction}
\label{sec:introduction}

Character traits, cognitive and non-cognitive abilities have been increasingly
embedded in economic models during the last one or two decades. In this paper
we focus on one specific personality trait, the locus of control, which goes
back to the work of \citet{rotter1966}. The locus of control refers to the
causality between an individual's behavior and its consequences. Individuals
with a rather internal locus of control think that they can entirely influence
the outcome of their behavior and actions whereas individuals with a more
external locus of control are of the opinion that fate or luck mainly determine
what happens to them in life. The locus of control is especially relevant in
economic applications as it is a major determinant of a wide range of life
outcomes.

According to \citet{heineck2010}, having a more internal locus of control is
associated with higher wages. \citet{piatek2010} discover that education is the
main channel for this phenomenon as individuals with an internal locus of
control invest more in education. \citet{gallo2003} find that unemployment
durations following a job loss are also lower for these people. Besides,
\citet{cobb2014} observe that individuals with an internal locus of control are
more likely to eat healthier and exercise regularly. Another benefit of having
an internal locus of control is found by \citet{buddelmeyer2016} who discover
that an internal locus of control increases the individual's psychological
resilience against certain negative shocks. In the literature, adults' locus of
control is often regarded as being stable or even fixed over time
\citep{heineck2010,semykina2007}. By treating it as exogenous, reverse
causality problems are overcome as it is assumed not to be influenced by any
labour market outcomes. This assumption is convenient if the locus of control
is measured only once in the data, but the results are likely to be biased if
the locus of control is indeed not stable over the relevant time frame. For an
elaborate explanation of the econometric issues that can arise, see
\citet{cobb2013}.

In this study, we want to challenge this assumption by investigating whether an
individual's locus of control changes after he or she has experienced an
exogenously caused traumatic event (e.g. the death of a family member).

This study is in line with the early work of \citet{alesina2002} who look at
changes in trust due to traumatic events. An extension of the framework to
locus of control was done by \citet{cobb2013} using Australian data
(HILDA survey) who find that positive and negative life events have only a
minor and empirically irrelevant  influence on the locus of control.
\citet{preuss2017} give a brief overview of the studies focusing on the
relationship between unemployment and locus of control. Using the GSOEP,
\citet{preuss2017} provide evidence with greater external validity. By
differentiating between stated locus of control, which is context-specific, and
actual locus of control, they do not reject the stability assumption of actual
locus of control. This is due to the fact that they observe a change in the
stated locus of control during unemployment which vanishes after reemployment.

Instead of focusing only on unemployment, we look at traumata concerning three
categories of life - the health, the social and the economic environment.  For
this purpose, we make use of the German Socio-Economic Panel \citep{gsoep2017}
and run three different regression specifications. By using a differences-in-
differences approach, we expect all of the traumatic events to lead to a shift
towards a more external locus of control.

The outline of the paper is as follows. Section~\ref{sec:data} describes the
data and shows possible causal channels of locus of control.
Section~\ref{sec:estimation_strategy} explains the estimation strategy and
Section~\ref{sec:results} presents the result. We provide further ideas to
extend this project in Section~\ref{sec:extensions} and conclude in
Section~\ref{sec:conclusion}.

\section{Data}
\label{sec:data}

Our data comes from the German Socio-Economic Panel \citep{gsoep2017} which is
an annual representative household panel study. It focuses on all participants
from 2005-2015 which is about 16,000 households or 30,000 people.

\subsection{Locus of Control} % (fold)
\label{sub:locus_of_control}

Our main sample restriction is the elicitation of the locus of control survey
questions which are asked 2005, 2010 and 2015. The questions are part of the
Personal Questionnaire. In these waves the survey participants have been asked
to what extent they agree or disagree with the statements presented in
Table~\ref{tab:loc_items} (1 = disagree completely; 7 = agree completely).

\begin{table}[H]
    \begin{tabular}{p{1.5cm} p{14cm}}
    \toprule
    Question: & \textit{The following statements apply to different attitudes
    towards life and the future. To what degree to you personally agree with
    the following statements? Please answer according to the following scale: 1
    means disagree completely, and 7 means agree completely.} \\
    \midrule
    1  & How my life goes depends on myself \\
    2  & Compared to others, I haven't achieved what I deserved \\
    3  & What one achieves in life is mainly a question of luck or fate \\
    4  & If a person is socially or politically active, he/she can have an effect on social conditions \\
    5  & I often have the experience that others make decisions regarding my life \\
    6  & One has to work hard in order to succeed \\
    7  & When I encounter difficulties, I have doubts about my abilities \\
    8  & The opportunities I have in life are determined by social conditions \\
    9  & Inborn abilities are more important than any efforts one can make \\
    10 & I have little control over the things that happen in my life \\
    \bottomrule
    \end{tabular}
    \caption{Locus of Control - Items, Source: GSOEP 2005, 2010, 2015}
    \label{tab:loc_items}
\end{table}

Following the specification of \citet{specht2013}, we only focus on items 1, 2,
3, 5, 7, 8, 10.\footnote{Items 4 and 6 cover cultural aspects and the wording
of item 9 cannot be unambiguously attributed to internal or external LoC.} We
also recoded all items so that higher numbers reflect a greater feeling of
control. A calculation of Cronbach's Alpha, a metric for the internal
consistency of a scale, yields a value of 0.698 for the seven item scale. This
is an improvement of 0.091 compared to the same metric applied to the ten item
scale. The measure can yield values in $]-\infty, 1]$ where values higher than
0.65 are acceptable as a rule of thumb.\footnote{The values of Cronbach's Alpha
are comparable the values reported in \citet{richter2017}.}

As \citet{piatek2016} find that using an index to construct a measure of locus
of control leads to measurement error and attenuation bias, we conduct a factor
analysis.\footnote{We also used PCA to extract locus of control, but found out
that the explained variance by the first principal component is about 40\% and
therefore insufficiently high. For completeness, Figure~\ref{fig:pca_ev}
reports the explained variance in percentages.}

One issue with factor analysis is that it requires observations to be
independent and identically distributed. As we have a panel dataset, the
assumption that observations are independent does not hold. Furthermore, we do
not want to employ multiple factor analyses for each wave in 2005, 2010 and
2015 since estimation would return different factor loadings which makes
comparison over time even more problematic. Our solution to this problem is to
fit the factor analysis on the LoC variables only in 2010. The advantage is
that independence is guaranteed and we fit the model on every individual since
every individual has answered the Personal Questionnaire in 2010.

Figure~\ref{fig:stated_loc} shows the resulting distribution of LoC over all
waves. The distribution is roughly similar to a standard normal distribution
indicated by the orange colored line. The explained variance ratio is displayed
in Figure~\ref{fig:fa_ev} and shows that the first factor explains more than
80\% of the variance. The factor loadings of the items are shown in
Figure~\ref{fig:fa_loadings}.

\begin{figure}[H]
    \centering
    \includegraphics[width=0.7\textwidth]{../../out/figures/fig-stated-loc-fa}
    \caption{Distribution of first factor of LoC for years 2005, 2010 and 2015}
    \label{fig:stated_loc}
\end{figure}

% subsection locus_of_control (end)


\subsection{Events} % (fold)
\label{sub:events}

To explore the malleability of locus of control, we identify several events
which might push the individual towards a more external locus of control. The
reasoning is that an exogenous shock increases the salience of lack of power
which changes the individual's belief of control over one's life. A full list
of events is shown in Table~\ref{tab:negative_life_events}. Another feature of
the events is that they must be sufficiently involuntary since we want to
eliminate the influence of selectivity and reverse causality.

We assume that these life events are negative, meaning they shift the locus of
control towards an external one. This characterization is not unambiguously as
even those unfortunate events can have a positive effect. The relatives of
person who died after a long time of a severe illness might feel relief due to
the loss of pressure a case of illness can cause. Another example is an
unplanned pregnancy which might be an involuntary but fortunate incident.

\begin{table}[H]
    \caption{List of Negative Life Events}
    \label{tab:negative_life_events}
    \centering

    \begin{tabular}{>{\quad}l}
    \toprule
    Questions\\
    \midrule
    \rule{0pt}{2.5ex}\rowgroup{\textit{Social Environment}} \\
    Child has Disorder\\
    Death of Child\\
    Death of Father\\
    Death of Household Person\\
    Death of Mother\\
    Death of Partner\\
    Divorce\\
    Household composition change\\
    Separation\\
    \rule{0pt}{2.5ex}\rowgroup{\textit{Economic Environment}} \\
    Displacement\\
    \rule{0pt}{2.5ex}\rowgroup{\textit{Health Environment}} \\
    Legally Handicapped\\
    Unplanned Pregnancy\\
    \bottomrule
    \end{tabular}
\end{table}

One important problem in the preparation of the events is the exact attribution
to the periods between two interviews. All events related to death,
displacement, divorce, household composition change, separation can be found in
the Personal Questionnaire with the exact month of the event in the survey or
previous year.

The information about unplanned pregnancies and whether a child is born with a
disorder can be found in the Mother and Child Questionnaire which follows
mothers and children from birth to the child's age of ten. Since there is no
monthly information about the events available, we had to make two assumptions.
First, unplanned pregnancies are connected to the primary month of the
pregnancy as pregnancies are usually discovered in the first two months.
Second, the month of birth is used as the temporal information for a child with
an disorder. There are multiple reasons for disorders, e.g. motoric, mental
disadvantages, chronic illnesses.

The degree of disability is asked yearly in the Personal Questionnaire but
without temporal information. We compute the change in disability and use only
changes from 0 to 100 percent as an event. We exclude negative changes, meaning
the transition to lower percentages as well as small changes which might be due
to different classifications of illnesses or attempts by individuals to receive
higher benefits. To be legally handicapped does not only mean to live without
one leg but also other forms of health shocks like cancer affect the degree of
disability. Therefore, this variable serves more as proxy for serious health
shocks.

As selectivity and anticipation plays a critical role in the assessment of the
malleability of locus of control, we create an additional variable for the
second regression specification which restricts displacement to job losses due
to displacement by employer and plant closure. Other reasons for job loss like
mutual agreement to terminate the contract are discarded. With this approach we
follow \citet{preuss2017}.

The final sample consists of 21,246 observations, 12191 individuals from
2005-2010 and 9235 individuals from 2010-2015. There are 7,206 people observed
over the whole time frame from 2005-2015 which allow for more extensive
analyses which are outlined in following sections. We have data on 14220 unique
individuals. Descriptive statistics of the controls comparing individuals with
and without events before the event are shown in the appendix (see
Table~\ref{tab:descriptive-statistics-any} to
Table~\ref{tab:descriptive-statistics-separated}).
Figure~\ref{fig:events_count} shows how often each event occurred per
individual between 2005-2010 and 2010-2015.

Now, we can compute the change in locus of control per observation between two
interviews and compare the average change for observations experiencing an
event and those who do not. Figure~\ref{fig:level-change-loc-event} shows the
resulting averages. The estimation of the effects and the quantification of
uncertainty surrounding these estimates will addressed in
Section~\ref{sec:results}.

\begin{figure}[H]
    \centering
    \includegraphics[width=\textwidth]{../../out/figures/fig-level-change-loc-event}
    \caption{Average change in Locus of Control for events}
    \label{fig:level-change-loc-event}
\end{figure}
% subsection events (end)

\subsection{Controls} % (fold)
\label{sub:controls}

Regarding the controls, we follow the specifications by \citet{cobb2013}.
Dummies for age groups are included such that each age group encompasses about
ten years. Marital status is a dummy for single or in relationship and is
derived from the legal status of the relationship. We control for immigrant
status and employment status, the latter can be employed, not employed or
others. Besides, we include seven education groups which range from no
education to higher education. Log household income and gender is added as
controls. Another covariate for the first and second regression specification
is the time passed in months since the last event occurred.

% subsection controls (end)


\section{Estimation Strategy}
\label{sec:estimation_strategy}

For investigating the impact of traumata on the locus of control, we conduct
three specifications, a more general, a more specific one and one including
just the number of traumata experienced. In all specifications we use change in
locus of control as a dependent variable and estimate the effect by a
differences-in-differences analysis.

In our first and second specification, we include not only the traumatic
events, but also interaction terms between the events and the time in months
between the occurrence of the event and the locus of control measurement.
Thereby, the coefficient of the event measures the effect of the event if it
happened right now and the coefficient of the interaction term measures the
effect of one additional month which has passed since.

\citet{preuss2017} find that the effect of a displacement on the locus of
control almost vanishes with reemployment. Consequently, we should also account
for the current situation for some events. These are not only displacements,
but also separations and divorces. We suppose that the effect on change in
locus of control of the latter depend a lot on whether the person is still
single or has found a new partner. Thus, we interact displacements with current
employment status and separations and divorces with current marital status.
Current marital status is only a proxy for current relationship status here, as
we do not observe the latter. By using these interaction terms, we investigate
the effect of the event separately, for the employed, unemployed or those with
another employment status or for those who are single and those in a
relationship.

The events we include vary in our specifications. In the first more general
specification of our analysis, we include the birth of a child who is
handicapped and a dummy which is one if the degree of disability of a household
member has changed from 0 to 100\%. On top of that, we use the occurrence of a
divorce, an unplanned pregnancy, a relationship breakup, the death of a
partner, father, mother, child or a household member or any other family
composition change. Besides, the loss of a job would be included in our first
specification.

In our second, more precise specification, we drop the occurrence of a divorce
and of a relationship break-up as we suppose that the latter incidents are in
most cases not exogenous and would therefore not necessarily lead to a more
external locus of control. On top of that, we include only the job losses which
are rather exogenous e.g. when the place of work closed or the person was
dismissed by his or her employer, but not if there was e.g. a mutual agreement
with the employer or the job was ended due to retirement.

In order to find out whether the relationship between change in locus of
control and negative life events varies with the number of negative life events
reported, we include the total number of traumas in our third specification
instead of using all the dummies for the different traumatic experiences.
Additionally, we include the total number squared to test for any non-
linearities. In this specification, we do not control for the time between the
occurrences of the traumata and the locus of control measurement.

In our first specification, we estimate the following equation

\begin{center}
    $ \Delta LOC_{i,t}= \alpha + X_{i,t} * \beta  + X_{i,t} ' T_{i,t} * \delta + W_{i,t} ' S_{i,t} * \gamma + (W_{i,t}' S_{i,t}) ' T_{i,t} * \zeta + Z_{i,t} * \eta + \epsilon_{i}  $
\end{center}

where $\alpha$ is the average change in locus of control, $
X\textsubscript{i,t}$ a vector of the trauma dummies and $ X
\textsubscript{i,t} ' T \textsubscript{i,t}$ comprises the interaction terms
between the traumata and the time between the occurrence of each individual
trauma and the locus of control measurement. $ W \textsubscript{i,t}$ is a
vector of the three traumatic events: divorce, separation and displacement, the
latter is included twice, and $ S \textsubscript{i,t} $ is a vector including a
dummy for Single (twice), for Not Employed and for Other Employment Status. $(W
\textsubscript{i,t} ' S \textsubscript{i,t}) ' T \textsubscript{i,t} $ is its
interaction with the times since the occurrence of the three traumatic events.
$ Z \textsubscript{i,t}$ is a vector in which the control variables are
included and $\epsilon\textsubscript{i}$ is the error term.

For all events apart from divorce, separation and displacement, the vector
$\beta$ measures the average treatment effect on the treated which shows how
each traumatic experience affects the change in locus of control on average
without any time effects. We try to select the most exogenous traumatic events
as possible, but the probability that someone suffers from an event differs
between the individuals. Older people are more likely to suffer from the death
of a father or a mother than younger people are or people who are single are
less likely to get an unplanned child than people in a relationship. As we are
not able to control for everything, we cannot say that we measure the average
treatment effect, but only the effect on the treated.

In our third specification, the estimated equation is

\begin{align}
    \Delta LOC_{i,t}= \alpha + N_{i,t} * \beta + N^{2}_{i,t} * \eta + Z_{i,t} *
    \gamma + \epsilon_{i}
\end{align}

in which we include the number of traumatic experiences $N \textsubscript{i,t}$
and the number of traumatic experiences squared $N
\textsuperscript{2}\textsubscript{i,t}$.

As our data is a panel spanning two time periods, we face the problem of
observing the same individual over time. In order to control for this
correlation, we would like to include individual fixed effects. Unfortunately,
this is not possible as we observe some individuals only for one period. It
would be interesting to see whether reducing the sample to only those
individuals which we observe for two periods and including individual fixed
effects for them would change our results.

For accounting for the fact that the residuals of the individuals are
correlated over time, we use clustered standard errors on the individual level.
The residuals of the households are also correlated over time as traumatic
experiences affect the entire household in most cases. Clustering on the
household level is problematic as the individuals change households and we
would need to cluster always on the household which the individual was part of
at the time of the trauma happening. As we include several traumata in our
regression which all happen at different times, this is not possible.

\section{Results} % (fold)
\label{sec:results}

The evidence from our first regression specification is mixed. Most of the
events affect the change in locus of control negatively, but none of the
effects is significant. The additional time effects have positive and negative
effects, but only the additional months passed since the death of a partner and
since a displacement are significantly different from zero. The longer the time
between the locus of control measurement and these two events, the greater the
change in the change in locus of control due to these events. While the effect
of a displacement itself is negative for the baseline group, the employed, it
is positive for those who are not employed and those who have another
employment status. Nevertheless, the effects are not significant. The effects
of a separation and a divorce also have the opposite sign for the baseline
group, those in a relationship, compared to those who are single. Again, the
effects are not significantly different from zero.

In our second more precise specification, the pattern is mixed as in our first
specification, but the time effects mentioned above are not significantly
different from zero anymore. Compared to \citet{preuss2017}, we do not find any
significant effects for those who experienced a displacement and who are
currently unemployed.

\begin{table}[H]
    \tiny{\begin{center}
 \begin{tabular}{lclc}
 \toprule
 \textbf{Dep. Variable:}                                                                                              & Change in Locus of Control & \textbf{  R-squared:         } &     0.005   \\
 \textbf{Model:}                                                                                                      &        OLS         & \textbf{  Adj. R-squared:    } &     0.003   \\
 \textbf{Method:}                                                                                                     &   Least Squares    & \textbf{  F-statistic:       } &     1.846   \\
 \textbf{Date:}                                                                                                       &  Thu, 25 Jan 2018  & \textbf{  Prob (F-statistic):} &  3.71e-05   \\
 \textbf{Time:}                                                                                                       &      13:10:54      & \textbf{  Log-Likelihood:    } &   -26849.   \\
 \textbf{No. Observations:}                                                                                           &        21371       & \textbf{  AIC:               } & 5.380e+04   \\
 \textbf{Df Residuals:}                                                                                               &        21320       & \textbf{  BIC:               } & 5.421e+04   \\
 \textbf{Df Model:}                                                                                                   &           50       & \textbf{                     } &             \\
 \bottomrule
 \end{tabular}
 \begin{tabular}{lcccccc}
                                                                                                                      & \textbf{coef} & \textbf{std err} & \textbf{t} & \textbf{P$>$$|$t$|$} & \textbf{[0.025} & \textbf{0.975]}  \\
 \midrule
 \textbf{Intercept}                                                                                                   &      -0.0319  &        0.054     &    -0.595  &         0.552        &       -0.137    &        0.073     \\
 \textbf{Child Has Disorders}                                                                           &       0.0988  &        0.113     &     0.877  &         0.381        &       -0.122    &        0.320     \\
 \textbf{Death of Child}                                                                              &      -0.1382  &        0.091     &    -1.513  &         0.130        &       -0.317    &        0.041     \\
 \textbf{Death of Father}                                                                             &       0.0211  &        0.038     &     0.549  &         0.583        &       -0.054    &        0.096     \\
 \textbf{Death of HH Person}                                                                         &      -0.0449  &        0.094     &    -0.476  &         0.634        &       -0.230    &        0.140     \\
 \textbf{Death of Mother}                                                                             &      -0.0216  &        0.039     &    -0.555  &         0.579        &       -0.098    &        0.055     \\
 \textbf{Death of Partner}                                                                            &      -0.0493  &        0.054     &    -0.913  &         0.361        &       -0.155    &        0.056     \\
 \textbf{Divorce}                                                                                  &      -0.2960  &        0.276     &    -1.072  &         0.284        &       -0.837    &        0.245     \\
 \textbf{HH Composition Changed}                                                                          &      -0.0285  &        0.056     &    -0.505  &         0.614        &       -0.139    &        0.082     \\
 \textbf{Displacement}                                                                          &      -0.0222  &        0.032     &    -0.693  &         0.488        &       -0.085    &        0.041     \\
 \textbf{Legally Handicapped}                                                                &      -0.1187  &        0.093     &    -1.280  &         0.200        &       -0.301    &        0.063     \\
 \textbf{Unplanned Pregnancy}                                                                      &      -0.0525  &        0.065     &    -0.808  &         0.419        &       -0.180    &        0.075     \\
 \textbf{Separation}                                                                                 &       0.1072  &        0.177     &     0.607  &         0.544        &       -0.239    &        0.453     \\
 \textbf{Divorce * Single}                                                     &       0.3002  &        0.326     &     0.920  &         0.358        &       -0.340    &        0.940     \\
 \textbf{Displacement * Not Employed}                                    &      -0.0142  &        0.031     &    -0.454  &         0.650        &       -0.076    &        0.047     \\
 \textbf{Displacement * Other Empl. Status}                                           &      -0.0258  &        0.058     &    -0.444  &         0.657        &       -0.140    &        0.088     \\
 \textbf{Separation * Single}                                                    &      -0.0988  &        0.212     &    -0.465  &         0.642        &       -0.515    &        0.318     \\
 \textbf{Child Disorders * Time Since Occurrence}                                          &      -0.0060  &        0.004     &    -1.489  &         0.136        &       -0.014    &        0.002     \\
 \textbf{Child Death * Time Since Occurrence}                                                &      -0.0010  &        0.006     &    -0.149  &         0.882        &       -0.013    &        0.012     \\
 \textbf{Father Death * Time Since Occurrence}                                              &      -0.0025  &        0.001     &    -1.705  &         0.088        &       -0.005    &        0.000     \\
 \textbf{Household Person Death * Time Since Occurrence}                                      &       0.0067  &        0.005     &     1.236  &         0.217        &       -0.004    &        0.017     \\
 \textbf{Mother Death * Time Since Occurrence}                                              &       0.0021  &        0.002     &     1.342  &         0.179        &       -0.001    &        0.005     \\
 \textbf{Partner Death * Time Since Occurrence}                                            &       0.0043  &        0.003     &     1.586  &         0.113        &       -0.001    &        0.010     \\
 \textbf{Divorce * Time Since Occurrence}                                                        &       0.0126  &        0.011     &     1.154  &         0.248        &       -0.009    &        0.034     \\
 \textbf{Divorce * Single * Time Since Occurrence}                           &      -0.0087  &        0.011     &    -0.775  &         0.438        &       -0.031    &        0.013     \\
 \textbf{HH Composition Changed * Time Since Occurrence}                                        &      -0.0011  &        0.003     &    -0.381  &         0.703        &       -0.007    &        0.005     \\
 \textcolor{red}{\textbf{Displacement * Time Since Occurrence}}                                        &       \textcolor{red}{0.0024}  &        \textcolor{red}{0.001}     &     \textcolor{red}{2.547}  &         \textcolor{red}{0.011}        &        \textcolor{red}{0.001}    &        \textcolor{red}{0.004}     \\
 \textbf{Displacement * Time Since Occurrence * Not Employed}  &      -0.0018  &        0.001     &    -1.270  &         0.204        &       -0.005    &        0.001     \\
 \textbf{Displacement * Time Since Occurrence * Other Empl. Status}         &      -0.0005  &        0.003     &    -0.190  &         0.849        &       -0.006    &        0.005     \\
 \textbf{Handicapped * Time Since Occurrence}                    &       0.0025  &        0.005     &     0.528  &         0.597        &       -0.007    &        0.012     \\
 \textbf{Unplanned Pregnancy * Time Since Occurrence}                                &       0.0014  &        0.003     &     0.440  &         0.660        &       -0.005    &        0.008     \\
 \textbf{Separation * Time Since Occurrence}                                                      &      -0.0029  &        0.008     &    -0.375  &         0.708        &       -0.018    &        0.012     \\
 \textbf{Separation * Time Since Occurrence * Single}                         &       0.0036  &        0.008     &     0.456  &         0.649        &       -0.012    &        0.019     \\
 \end{tabular}
 %\caption{OLS Regression Results}
 \end{center}}
    \footnotesize \textit{Included controls: education, age, ln(hh income),
    employment status, migration status, gender, marital status; Standard
    errors are clustered on the individual level.}
    \caption{First Specification}
    \label{tab:first_reg}
\end{table}


\begin{table}[H]
    \tiny{\input{../../out/tables/reg_table_2.tex}}
    \footnotesize \textit{Included controls: education, age, ln(hh income),
    employment status, migration status, gender, marital status; Standard
    errors are clustered on the individual level.}
    \caption{Second Specification}
    \label{tab:second_reg}
\end{table}

\begin{table}[H]
    \input{../../out/tables/reg_table_3.tex}
    \footnotesize \textit{Included controls: education, age, ln(hh income),
    employment status, migration status, gender, marital status; Standard
    errors are clustered on the individual level.}
    \caption{Third Specification}
    \label{tab:third_reg}
\end{table}

In our third specification, we see that the number of traumata experienced
increases the change in the locus of control with a decreasing rate, but this
effect is not significant either.

% section results (end)

\section{Extensions} % (fold)
\label{sec:extensions}

In this section we want to highlight some extensions of our research for which
we had not enough time.

\subsection{Internal vs. External types} % (fold)
\label{sub:internal_vs_external_types}

Evidence from \citet{buddelmeyer2016} suggests that having an internal locus of
control can insure against emotional pain caused by an event like illness,
close family member detained in jail, becoming a victim of crime and death of a
close friend. We could also use our dataset to analyze different effects of
traumata across the distribution of locus of control in the sample.

First, we split the sample by the median locus of control and categorize
individuals lower than the median as Externals and above the median as
Internals. Second, we could run our previous regression models including a
dummy for the locus of control type and interaction terms with events.

% subsection internal_vs_external_types (end)

\subsection{Time Effects} % (fold)
\label{sub:time_effects}

In this section we want to improve on our results regarding events and time
effects. In the regression specification 1 and 2, we include time variables
which represent the time in month since the last event to control for
differences in time between events and locus of control elicitations. The
channel of time regarding traumatic events is ambiguous as there are two
possible fringe cases. First, the salience of the traumatic event fades as time
passes and the effect on locus of control might wash out. For example, the
death of a close relative becomes more irrelevant as an individual proceeds
with his daily life and overcomes the sadness. Second, salience may exacerbate
with more time as the conflict is still unresolved. When a person looses her
job and does not find immediate reemployment, the doubts about her own ability
might increase.

Our former estimation sample might not be sufficiently specified to assess this
problem. As we have restricted our sample to the period 2005-2015, we have not
captured information about events before 2005 or 2010 if a person enters the
sample in 2005 or 2010, respectively. To overcome this problem with our data,
we analyze only individuals which are observed from 2005 to 2015. This group
comprises 14412 observations. Next, we restrict the group to all observations
who experienced no traumatic event in the first period of 2005-2010. Our final
sample contains 4191 observations. At last, we modify the second regression
specification by splitting the continuous time variable for each event into
seven dummies where the first group contains all individuals without an event
and the other six groups cover a twelve month interval each.

The coefficients for the interaction terms between event and time intervals
allow to test for diminished or increased effects over time for different
events.

% subsection time_effects (end)

\subsection{Test for Common Trends Assumption} % (fold)
\label{sub:test_for_common_trends_assumption}

As we are using a differences-in-differences approach, we rely on the
assumption of parallel trends between control and treatment group in absence of
the treatment. We want to employ an approach similar to \citet{autor2003} and
specify our dataset the following way. First, select all individuals which are
observed from 2005-2015 ($7206 * 2$ observations). Second, we will focus on
individuals which have no event in the first period from 2005-2010 and any
number of events in the second. Our final sample consists of $4191 * 2$
observations.

The test whether the treatment group changes before receiving the treatment is
to include dummies into the regression called leads. They precede the actual
treatment for multiple periods. If these coefficients turn out to be
significant, the common trend assumption does not hold.

Regarding our data, we indicate for each observation whether an event will
occur in the next period. We could use a dummy variable indicating whether any
event will happen, dummy variables for each event to differentiate between
treatments or a continuous variable representing the number of months until the
event will happen.

% subsection test_for_common_trends_assumption (end)

% section extensions (end)

\section{Conclusion} % (fold)
\label{sec:conclusion}

Using three different specifications, we investigated the assumption of a
stable locus of control using the German Socio-Economic Panel. We analyzed how
traumatic experiences such as a displacement or an unplanned pregnancy change
the locus of control and find that none of the events changes it significantly.
We also tested whether the overall number of traumatic events experienced in
the previous five years changes it and do not find a significant effect either.
Our results are in line with those from \citet{cobb2013} who do not find any
effects of traumata on the locus of control using Australian data.

Furthermore, we provide several useful extensions for our analysis which would
improve this research project. Other extensions are for future research.

We only focused on negative life events, but in future research it would be
interesting to see the effect of positive life events on the locus of control
in addition.

\citet{specht2013} find that the locus of control does not remain stable for
adults, but varies with age. Related to this finding, there could be age
periods in which the locus of control is more stable and others in which it is
easily affected by traumatic events. Extending our study, one could compare the
effects of the traumatic events between different age groups in order to see
whether the effects are greater for certain groups.

% section conclusion (end)

\printbibliography

\appendix
\setcounter{secnumdepth}{0}
\section{Appendix}

\subsection{Figures} % (fold)
\label{sub:figures}

\begin{figure}[H]
    \centering
    \includegraphics[width=0.7\textwidth]{../../out/figures/fig-fa-two-comp-explained-variance}
    \caption{Factor Analysis - Explained Variance}
    \label{fig:fa_ev}
\end{figure}

\begin{figure}[H]
    \centering
    \includegraphics[width=0.7\textwidth]{../../out/figures/fig-pca-two-comp-explained-variance}
    \caption{Principal Component Analysis - Explained Variance}
    \label{fig:pca_ev}
\end{figure}

\begin{figure}[H]
    \centering
    \includegraphics[width=0.7\textwidth]{../../out/figures/fig-fa-factor-loadings}
    \caption{Factor Analysis - Factor Loadings}
    \label{fig:fa_loadings}
\end{figure}

\begin{figure}[H]
    \centering
    \includegraphics[width=\textwidth]{../../out/figures/fig-events-count}
    \caption{Number of events of individuals during 2005-2010 and 2010-2015}
    \label{fig:events_count}
\end{figure}


% subsection figures (end)

\subsection{Descriptive Statistics} % (fold)
\label{sub:descriptive_statistics}

The following tables present descriptives statistics for control variables for
the group who experiences the specific event and the group who does not before
the treatment.

\input{../../out/tables/tab-panel-descriptive-statistics-any}
\input{../../out/tables/tab-panel-descriptive-statistics-child-disorder}
\input{../../out/tables/tab-panel-descriptive-statistics-death-child}
\input{../../out/tables/tab-panel-descriptive-statistics-death-father}
\input{../../out/tables/tab-panel-descriptive-statistics-death-hh-person}
\input{../../out/tables/tab-panel-descriptive-statistics-death-mother}
\input{../../out/tables/tab-panel-descriptive-statistics-death-partner}
\input{../../out/tables/tab-panel-descriptive-statistics-divorced}
\input{../../out/tables/tab-panel-descriptive-statistics-hh-comp-change}
\input{../../out/tables/tab-panel-descriptive-statistics-last-job-ended}
\input{../../out/tables/tab-panel-descriptive-statistics-last-job-ended-limited}
\input{../../out/tables/tab-panel-descriptive-statistics-legally-handicapped-perc}
\input{../../out/tables/tab-panel-descriptive-statistics-pregnancy-unplanned}
\input{../../out/tables/tab-panel-descriptive-statistics-separated}

% subsection descriptive_statistics (end)

\end{document}
