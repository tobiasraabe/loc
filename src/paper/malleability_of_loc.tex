\documentclass[12pt, a4paper, fleqn, parskip]{scrartcl}
\def\datum{\today}
\author{Tobias Raabe and Imke Strampe}
\title{Malleability of Locus of Control by Traumatic Events}

\usepackage[english]{babel}  % set language to adjust headings, automatic hypenation
\usepackage[utf8]{inputenc}  % allows to use umlauts and certain symbols
\usepackage[T1]{fontenc}  % typesetting, necessary for umlauts and accentuations
\usepackage{lmodern}  % Latin Modern Fonts are successors of standard fonts
\usepackage{amsmath, amssymb, amsthm}
\usepackage{booktabs}
\usepackage{graphicx}
\usepackage[space]{grffile}
\usepackage{hyperref}
\usepackage{setspace}
\usepackage{xcolor}
\hypersetup{
    colorlinks,
    linkcolor={red!50!black},
    citecolor={blue!50!black},
    urlcolor={blue!80!black}
}

\usepackage[backend=biber, natbib=true, bibencoding=utf8, safeinputenc=true,
bibstyle=authoryear, citestyle=authoryear-comp, maxcitenames=2, mincitenames=1,
uniquelist=false, useprefix=true, maxbibnames=99, minbibnames=99, backref=true,
backrefstyle=three, doi=true, isbn=false, sortcites=false, dashed=false,
firstinits=true]{biblatex}

\addbibresource{references.bib}

% \documentclass[12pt,a4paper,fleqn]{article}
% \usepackage[english]{babel}
% %\usepackage[utf8]{inputenc}
% \usepackage{ae} % falls in acrobat reader nicht schön dargestellt werden kann
% \usepackage{amsmath} % für mathematische symbole
% \usepackage{amsfonts}
% \usepackage{amssymb}
% \usepackage{graphicx}
% \usepackage{tabularx}
% \usepackage[left=3.0cm,top=2.0cm,right=2.0cm,bottom=2.0cm]{geometry}
% \usepackage{filecontents}
% \usepackage{booktabs}
% \usepackage{multicol}
% \usepackage{appendix}
% \usepackage{subfigure}
% \usepackage{supertabular}
% \usepackage{caption}
% \usepackage{array, multirow}
% \usepackage{lscape}
% \usepackage{fixltx2e}
% \usepackage[scaled]{uarial}
% \usepackage{placeins} %Fuer den FloatBarrier Befehl
% \usepackage[latin1]{inputenc}
% \usepackage{diagbox}
% \renewcommand{\baselinestretch}{1.44} % entspricht 1,5-fachem Zeilenabstand in Microsoft Word
% \usepackage{natbib} % Usepackage fuer Bibliographie
% \bibliographystyle{apalike} % Usepackage fuer Bibliographie
% \renewcommand{\footnotesize}{\scriptsize}

\pagenumbering{roman}
\begin{document}
	\begin{titlepage}
		\begin{center}
			\vspace*{35mm}
			{\bfseries\Large{Malleability of Locus of Control by Traumatic Events}\\
			\vspace{25mm}}
			\begin{spacing}{1.3}
				\large Research Module in Management and Applied Microeconomics\\
				\vspace{65mm}
			   Prof. Dr. Thomas Dohmen, Dr. Philipp Eisenhauer \\
				\vspace{10mm}
				January 2018   \\
				\textbf{Tobias Raabe, Imke Strampe} \\
			\end{spacing}
		\end{center}
	\end{titlepage}
\tableofcontents

\pagebreak

\listoffigures  \addcontentsline{}{List of Figures} \clearpage
\listoftables  \addcontentsline{}{List of Tables} \clearpage

\pagebreak

\pagenumbering{arabic}

\section{Introduction}
Character traits, cognitive and non-cognitive abilities have been increasingly
embedded in economic models during the last one or two decades. In this paper
we focus on one specific personality trait, the locus of control, which goes
back to the work of Rotter (1966). The locus of control refers to the causality
between an individual's behaviour and its consequences. Individuals with a
rather internal locus of control think that they can entirely influence the
outcome of their behaviour and actions whereas individuals with a more external
locus of control are of the opinion that fate or luck mainly determine what
happens to them in life. The locus of control is especially relevant in
economic applications as it is a major determinant of a wide range of life
outcomes.

According to Heineck and Anger (2010), having a more internal locus of control
is associated with higher wages. Piatek and Pinger (2010) discover that
education is the main channel for this phenomenon as individuals with an
internal locus of control invest more in education. Gallo et al. (2013) find
that unemployment durations following a job loss are also lower for these
people. Besides, Cobb-Clark et al. (2014) observe that individuals with an
internal locus of control are more likely to eat healthier and exercise
regularly. Another benefit of having an internal locus of control is found by
Buddelmeyer, Powdthavee (2016) who discover that an internal locus of control
increases the individual's psychological resilience against certain negative
shocks. In the literature, adults' locus of control is often regarded as being
stable or even fixed over time (see for example Heineck, Anger (2010),
Semykina, Linz (2007)). By treating it as exogenous, reverse causality problems
are overcome as it is assumed not to be influenced by any labour market
outcomes. This assumption is convenient if the locus of control is measured
only once in the data, but the results are likely to be biased if the locus of
control is indeed not stable over the relevant time frame. For an elaborate
explanation of the econometric issues that can arise, see Cobb-Clark, Schurer
(2013).

In this study, we want to challenge this assumption by investigating whether an
individual's locus of control changes after he or she has experienced an
exogenously caused traumatic event (e.g. the death of a family member).

This study is in line with the early work of Alesina, Ferrara (2002) who look
at changes in trust due to traumatic events. An extension of the framework to
locus of control was done by Cobb-Clark, Schurer (2013) using Australian data
(HILDA survey) who find that positive and negative life events have only a
minor and empirically irrelevant  influence on the locus of control. Preuss,
Hennecke (2017) give a brief overview of the studies focusing on the
relationship between unemployment and locus of control. Using the GSOEP,
Preuss, Hennecke (2017) provide evidence with greater external validity. By
differentiating between stated locus of control, which is context-specific, and
actual locus of control, they do not reject the stability assumption of actual
locus of control. This is due to the fact that they observe a change in the
stated locus of control during unemployment which vanishes after reemployment.

Instead of focusing only on unemployment, we look at traumatas concerning three
categories of life - the health, the social and the economic environment.  For
this purpose, we make use of the German Socio-Economic Panel (SOEP, 2016) and
run three different regression specifications. By using a differences-in-
differences approach, we expect all of the traumatic events to lead to a shift
towards a more external locus of control.

The outline of the paper is as follows. Section 2 describes the data and
analyses possible causal channels of locus of control. Section 3 explains the
estimation strategy and section 4 presents the result. We conclude with section
6.

\section{Data}

Our data comes from the German Socio-Economic Panel (SOEP) which is an annual
representative household panel study. It focuses on all participants from
2005-2015 which is about 16,000 households or 30,000 people.

Our main sample restriction is the elicitation of the locus of control survey
questions which are asked every five years from 2005 on. In these waves the
survey participants have been asked to what extent they agree or disagree with
the statements presented in Table 1 (1 = disagree completely; 7 = agree
completely).

\begin{table}[h!]
	\centering
	\begin{tabular}{ |p{1cm}||p{14cm}|  }
	\hline
	\multicolumn{2}{|c|}{The following statements apply to different attitudes towards life and the future.} \\
	\multicolumn{2}{|c|}{To what degree to you personally agree with the following statements?} \\
	\multicolumn{2}{|c|}{Please answer according to the following scale: 1 means disagree completely,}\\
	\multicolumn{2}{|c|}{and 7 means agree completely.}   \\
	\hline
	1   & How my life goes depends on me    \\
	2&   Compared to other people, I have not achieved what I deserve  \\
	3 & What a person achieves in life is above all a question of fate or luck\\
	4 & If a person is socially or politically active, he/she can have an effect on social conditions\\
	5& I frequently have the experience that other people have a controlling infuence over my life\\
	6 & One has to work hard in order to succeed\\
	7 & If I run up against difficulties in life, I often doubt my own abilities\\
	8 & Opportunities I have in life are determined by the social conditions\\
	9 & Inborn abilities are more important than any efforts one can make\\
	10 & I have little control over the things that happen in my life\\
	\hline
	\end{tabular}
	\caption{Source: SOEP 2005, 2010, 2015}
\end{table}

As Piatek, Pinger (2016) find that using an index to construct a measure of
locus of control leads to measurement error and attenuation bias, we conduct a
principal component analysis for getting a more valid measure.

For the traumatic experiences concerning health, we include a dummy which is
one if someone has born a child who has confirmed disorders and which is zero
otherwise. On top of that, we use the change in disability of a household
member in one specification and a dummy which is one only if the change is from
one to hundred percent in another specification.

Traumata relating to the social environment are a divorce dummy, a relationship
breakup dummy, an unplanned pregnancy dummy, a dummy for the death of a
partner, mother, father, child or a household member.

We also make use of a traumatic experience concerning the economic environment
which is a dummy for the loss of a job.

\textit{sample restrictions, missing values, how many people suffered from how
many traumata}

We expect to see a shift towards a more external locus of control after the
occurrence of each of the traumata. If such an exogenously caused, dreadful
incident happens to someone, the person might wonder why he or she was
affected. He might feel as a victim, become more doubtful and have the
impression that he does not deserve that. Besides, his feeling of having
control over what happens to him would shrink. Consequently he would start
believing more in fate than in his own ability of mastering everything and the
locus of control would shift towards externality. The entire process could
happen subconsciously.

\section{Estimation Strategy}

For investigating the impact of traumata on the locus of control, we conduct
three specifications, a more general, a more specific one and one including
just the number of traumata experienced. In all specifications we use change in
locus of control as a dependent variable and estimate the effect by a
differences-in-differences analysis. Regarding the controls, we follow the
specifications by Cobb-Clark, Schurer (2013): dummies for age groups, marital
status, immigrant status, employment status, educational qualification dummies
and log household income. Furthermore, we include a gender dummy and dummies
for years into the regression.

We would also like to control for the time passed since the last shock to
identify depreciation effects of traumatic events.

In the first more general specification of our analysis, we include the birth
of a child who is handicapped and the change in the degree of disability of a
household member. On top of that, we use the occurence of a divorce, an
unplanned pregnancy, a relationship breakup, the death of a partner, father,
mother, child or a household member or any other family composition change.
Besides, the loss of a job would be included in our first specification.

In our second, more precise specification, we include the change in the degree
of disability of a household member only in case of a change from 0 to 100\%.
Moreover, we drop the occurence of a divorce and of a relationship break-up as
we suppose that the latter incidents are in most cases not exogenous and would
therefore not necessarily lead to a more external locus of control. Regarding
deaths, we would drop deaths where the person investigated is not strongly
affected by the loss. On top of that, we include only the job losses which are
rather exogenous e.g. when the place of work closed or the person was dismissed
by his or her employer, but not if there was e.g. a mutual agreement with the
employer or the job was ended due to retirement.

In order to find out whether the relationship between change in locus of
control and negative life events varies with the number of negative life events
reported, we include the total number of traumas in our third specification
instead of using all the dummies for the different traumatic experiences.
Additionally, we include the total number squared to test for any non-
linearities. In this specification, we do not control for the time between the
occurrences of the traumata and the locus of control measurement.

In our first and second specification, we estimate the following equation

\begin{align}
	\Delta LOC \textsubscript{i,t}= \alpha + X \textsubscript{i,t} * \beta + Z
	\textsubscript{i,t} * \gamma + Y\textsubscript{t} * \delta +
	\epsilon\textsubscript{i}
\end{align}

where $\alpha$ is the average change in locus of control, $
X\textsubscript{i,t}$ a vector of the traumata dummies, $ Z\textsubscript{i,t}$
a vector in which the control variables are included, $ Y\textsubscript{t}$
comprises the year dummies and $\epsilon\textsubscript{i}$ is the error term.

The vector $\beta$ measures the average treatment effect on the treated which
shows how each traumatic experience changes the change in locus of control over
time on average. In our first specification, in which the traumata are not
necessarily exogenous, $\beta $ is clearly the effect only on the treated as
there could be some possible selection. In our second specification, we make
sure that all traumata are exogenously caused and therefore there should be no
selection in or out of treatment. Consequently, we would expect $\beta$ to
measure the average treatment effect without any conditioning.

In our third specification, the estimated equation is

\begin{align}
	\Delta LOC \textsubscript{i,t}= \alpha + N \textsubscript{i,t} * \beta + N
	\textsuperscript{2}\textsubscript{i,t} * \eta + Z \textsubscript{i,t} *
	\gamma + Y\textsubscript{t} * \delta + \epsilon\textsubscript{i}
\end{align}

in which we include the number of traumatic experiences $N \textsubscript{i,t}$
and the number of traumatic experiences squared $N
\textsuperscript{2}\textsubscript{i,t}$.

We use clustered standard errors on the individual level to account for
correlation between the same individual over time in all three specifications.

\printbibliography

\appendix
\setcounter{secnumdepth}{0}
\section{Appendix}




\pagebreak
\setcounter{secnumdepth}{0}
\section{References}




\pagebreak
\setcounter{secnumdepth}{0}
\section{Declaration}

\end{document}
