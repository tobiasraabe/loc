
\documentclass{beamer}

\renewcommand{\baselinestretch}{1.2}\normalsize
\usetheme{default}
\setbeamertemplate{navigation symbols}{}
\setbeamertemplate{footline}[frame number]

\date{January 25, 2018}
\author[Raabe and Strampe]{Tobias Raabe and Imke Strampe}
\title{Malleability of Locus of Control by Traumatic Events}
\institute[Uni Bonn]{University of Bonn}

\usepackage[english]{babel}  % set language to adjust headings, automatic hypenation
\usepackage[utf8]{inputenc}  % allows to use umlauts and certain symbols
\usepackage[T1]{fontenc}  % typesetting, necessary for umlauts and accentuations
\usepackage{lmodern}  % Latin Modern Fonts are successors of standard fonts
\usepackage{amsmath, amssymb, amsthm}
% Packages for tables
\usepackage{booktabs}
\usepackage{caption}
\usepackage{array}
\newcommand{\rowgroup}[1]{\hspace{-1em}#1}  % Allows for unindented groups in tables
\usepackage{float}

\usepackage{graphicx}
\usepackage[space]{grffile}
\usepackage{hyperref}
\usepackage{setspace}
\usepackage{xcolor}
\hypersetup{
    colorlinks,
    linkcolor={red!50!black},
    citecolor={blue!50!black},
    urlcolor={blue!80!black}
}

\usepackage{csquotes}
\usepackage[backend=biber, natbib=true, bibencoding=utf8, safeinputenc=true,
bibstyle=authoryear, citestyle=authoryear-comp, maxcitenames=2, mincitenames=1,
uniquelist=false, useprefix=true, maxbibnames=99, minbibnames=99, backref=true,
backrefstyle=three, doi=true, isbn=false, sortcites=false, dashed=false,
giveninits=true, uniquename=init]{biblatex}

\addbibresource{references.bib}

\begin{document}

\begin{frame}[t]

\titlepage

\end{frame}

\begin{frame}[t]\frametitle{Table of Contents}

\tableofcontents

\end{frame}

\section{Introduction} % (fold)
\label{sec:introduction}

\begin{frame}[t]\frametitle{Introduction}

\begin{itemize}[<+->]
    \item "Locus of Control is a generalizing trait describing the extent to which a person believes he or she has influence over his or her own life" \citep{specht2013}
    \item \citet{heineck2010} find wage penalties for women and men with external locus of control
    \item \citet{piatek2010} find that individuals with an internal LoC invest more in education
    \item \citet{gallo2003} find that unemployment durations are lower for individuals with internal LoC
\end{itemize}

\note[item]{high sense of control means the person believes to have a strong impact on the things in their lives}
\note[item]{needs to be distinguished from domain-specific control beliefs such as health loc or intellectual loc}
\note[item]{heineck: interpretation is that if I do not think that I have influence on my labor market success I am not willing to exert more effort}

\end{frame}

% section introduction (end)

\section{Data} % (fold)
\label{sec:data}

\begin{frame}[t]\frametitle{GSOEP}

\begin{itemize}
    \item data comes from the GSOEP version 32.1 (1984-2015)
    \item includes about 16,000 households and 30,000 individuals
    \item Locus of Control (LoC) is asked in 2005, 2010 and 2015 in the Personal
  Questionnaire
\end{itemize}

\note[item]{also asked in 1999 on a four point scale}

\end{frame}

\begin{frame}[c]\frametitle{Locus of Control - Items}

\begin{table}[H]\tiny
    \begin{tabular}{p{1.5cm} p{8cm}}
    \toprule
    Question: & \textit{The following statements apply to different attitudes
    towards life and the future. To what degree to you personally agree with
    the following statements? Please answer according to the following scale: 1
    means disagree completely, and 7 means agree completely.} \\
    \midrule
    1  & How my life goes depends on myself \\
    2  & Compared to others, I haven't achieved what I deserved \\
    3  & What one achieves in life is mainly a question of luck or fate \\
    4  & If a person is socially or politically active, he/she can have an effect on social conditions \\
    5  & I often have the experience that others make decisions regarding my life \\
    6  & One has to work hard in order to succeed \\
    7  & When I encounter difficulties, I have doubts about my abilities \\
    8  & The opportunities I have in life are determined by social conditions \\
    9  & Inborn abilities are more important than any efforts one can make \\
    10 & I have little control over the things that happen in my life \\
    \bottomrule
    \end{tabular}
    \caption{Locus of Control - Items, Source: GSOEP 2005, 2010, 2015}
    \label{tab:loc_items}
\end{table}

\note[item]{scale range from 1 "disagree completely" to 7 "agree completely"}
\note[item]{item 1 points to internal, item 10 to external}

\end{frame}

\begin{frame}[t]\frametitle{Locus of Control - Preparation}

\begin{enumerate}[<+->]
    \item recode items so that higher numbers indicate greater feeling of
    control (e.g. item 10)
    \item exclude items 4, 6 (relate to cultural attitudes) and 9 (ambiguous)
    \item Cronbach's Alpha raises from 0.607 to 0.698 while moving from ten to
    seven items
    \item extract locus of control with factor analysis from the 2010 wave
\end{enumerate}

\note[item]{Cronbach's Alpha is the average correlation between the items corrected by Spearman-Brown formula}
\note[item]{factor analysis requires observations to be iid, cannot use whole
panel}
\note[item]{use only 2010 instead, constant factor loadings, every individual
is covered}

\end{frame}

\begin{frame}[c]\frametitle{Locus of Control - Explained Variance Ratio}

\begin{figure}[H]
    \centering
    \includegraphics[width=\textwidth]{../../out/figures/fig-fa-two-comp-explained-variance}
    \caption{Factors obtained by factor analysis}
    \label{fig:fa-ev}
\end{figure}

\end{frame}

\begin{frame}[c]\frametitle{Locus of Control - Factor Loadings}

\begin{figure}[H]
    \centering
    \includegraphics[width=\textwidth]{../../out/figures/fig-fa-factor-loadings}
    \caption{Factor loadings of items}
    \label{fig:fa-loadings}
\end{figure}

\end{frame}

\begin{frame}[c]\frametitle{Locus of Control - Distribution}

\begin{figure}[H]
    \centering
    \includegraphics[width=\textwidth]{../../out/figures/fig-stated-loc-fa}
    \caption{Distribution of first factor of LoC for years 2005, 2010 and 2015}
    \label{fig:stated-loc-dist}
\end{figure}

\end{frame}

\begin{frame}[c]\frametitle{Events}\tiny

\begin{table}[H]
    \caption{List of Negative Life Events}
    \label{tab:negative_life_events}
    \centering

    \begin{tabular}{>{\quad}l}
    \toprule
    Questions\\
    \midrule
    \rule{0pt}{2.5ex}\rowgroup{\textit{Social Environment}} \\
    Child has Disorder\\
    Death of Child\\
    Death of Father\\
    Death of Household Person\\
    Death of Mother\\
    Death of Partner\\
    Divorce\\
    Household composition change\\
    Separation\\
    \rule{0pt}{2.5ex}\rowgroup{\textit{Economic Environment}} \\
    Displacement\\
    \rule{0pt}{2.5ex}\rowgroup{\textit{Health Environment}} \\
    Legally Handicapped\\
    Unplanned Pregnancy\\
    \bottomrule
    \end{tabular}
\end{table}

\note[item]{Regarding preparation, it was crucial to assign events to the correct period}
\note[item]{Unplanned pregnancies and child disorders were assigned to beginning month of pregnancy and month of birth}
\note[item]{Legally handicapped only captures jumps from 0 to 100 and is assigned to the interview month, wanted to capture severe health shocks as it is not only an amputated arm but also cancer}
\note[item]{All other events can be attributed to a specific month}
\note[item]{Displacement comes in two flavors, restricted to plant closure and displacement by employer}
\note[item]{Not all events can be unambiguously categorized as negative, death of old relative with severe chronic illness, unplanned pregnancy, etc.}
\note[item]{Controls are Age groups for decades, Marital status (single, in relationship), immigration status, employment status (employed, unemployed, other), education groups (five groups), log household income, gender}

\end{frame}

\begin{frame}[plain]

\begin{figure}[H]
    \centering
    \includegraphics[width=\textwidth, height=\textheight]{../../out/figures/fig-event-count}
    \caption{Number of events of individuals during 2005-2010 and 2010-2015}
    \label{fig:event_count}
\end{figure}

\end{frame}

\begin{frame}[t]\frametitle{Mean Analysis}

\begin{figure}[H]
    \centering
    \includegraphics[width=\textwidth]{../../out/figures/fig-level-change-loc-event}
    \caption{Average change in Locus of Control for events}
    \label{fig:level_change_loc_event}
\end{figure}

\note[item]{Shows the average change in locus of control for each event in comparison with all other observations not experiencing the event.}

\end{frame}

% section data (end)


\section{Estimation Strategy} % (fold)
\label{sec:estimation_strategy}

\begin{frame}[t]\frametitle{Estimation Strategy}
	\begin{itemize}
		\item<+-> dependent variable: change in locus of control
		\item<+-> difference-in-difference analysis
		\item<+-> specifications:
		\begin{enumerate}
			\item all negative life events
			\item all negative life events apart from divorce and separation and only exogenous job losses
			\item total number of traumata, total number of traumata squared
		\end{enumerate}
	\end{itemize}
\end{frame}

\begin{frame}[t]\frametitle{Estimation Strategy}
	\begin{itemize}
		\item<+-> specification 1 + 2:\\
			\begin{itemize}
			\setlength{\itemsep}{20pt}
			\item<+-> for all events apart from displacement (divorce, separation):\\
			{\centering
			\textit{eventdummy + eventdummy * time since occurrence}\\}
			\item<+-> for displacement (divorce, separation):\\
			{\centering
			\textit{eventdummy + eventdummy * time since occurrence + eventdummy * status + eventdummy * status * time since occurrence}\\}

		\begin{itemize}
			\item status for displacement: employment status
			\item status for divorce: marital status
			\item status for separation: marital status
		\end{itemize}
			\end{itemize}
\end{itemize}
\end{frame}

\begin{frame}[t]\frametitle{Estimation Strategy}
\begin{itemize}
	\item all specifications:\\
	\begin{itemize}
		\setlength{\itemsep}{20pt}
		\item controls: education, age, ln(hh income), employment status, migration status, gender, marital status
		\item clustered standard errors on the individual level
		\end{itemize}
	\end{itemize}
\end{frame}
% section estimation_strategy (end)

\section{Results} % (fold)
\label{sec:results}

\begin{frame}[plain] % plain
\fontsize{5}{5}\selectfont
\begin{table}
	\begin{center}
 \begin{tabular}{lclc}
 \toprule
 \textbf{Dep. Variable:}                                                                                              & Change in Locus of Control & \textbf{  R-squared:         } &     0.005   \\
 \textbf{Model:}                                                                                                      &        OLS         & \textbf{  Adj. R-squared:    } &     0.003   \\
 \textbf{Method:}                                                                                                     &   Least Squares    & \textbf{  F-statistic:       } &     1.846   \\
 \textbf{Date:}                                                                                                       &  Thu, 25 Jan 2018  & \textbf{  Prob (F-statistic):} &  3.71e-05   \\
 \textbf{Time:}                                                                                                       &      13:10:54      & \textbf{  Log-Likelihood:    } &   -26849.   \\
 \textbf{No. Observations:}                                                                                           &        21371       & \textbf{  AIC:               } & 5.380e+04   \\
 \textbf{Df Residuals:}                                                                                               &        21320       & \textbf{  BIC:               } & 5.421e+04   \\
 \textbf{Df Model:}                                                                                                   &           50       & \textbf{                     } &             \\
 \bottomrule
 \end{tabular}
 \begin{tabular}{lcccccc}
                                                                                                                      & \textbf{coef} & \textbf{std err} & \textbf{t} & \textbf{P$>$$|$t$|$} & \textbf{[0.025} & \textbf{0.975]}  \\
 \midrule
 \textbf{Intercept}                                                                                                   &      -0.0319  &        0.054     &    -0.595  &         0.552        &       -0.137    &        0.073     \\
 \textbf{Child Has Disorders}                                                                           &       0.0988  &        0.113     &     0.877  &         0.381        &       -0.122    &        0.320     \\
 \textbf{Death of Child}                                                                              &      -0.1382  &        0.091     &    -1.513  &         0.130        &       -0.317    &        0.041     \\
 \textbf{Death of Father}                                                                             &       0.0211  &        0.038     &     0.549  &         0.583        &       -0.054    &        0.096     \\
 \textbf{Death of HH Person}                                                                         &      -0.0449  &        0.094     &    -0.476  &         0.634        &       -0.230    &        0.140     \\
 \textbf{Death of Mother}                                                                             &      -0.0216  &        0.039     &    -0.555  &         0.579        &       -0.098    &        0.055     \\
 \textbf{Death of Partner}                                                                            &      -0.0493  &        0.054     &    -0.913  &         0.361        &       -0.155    &        0.056     \\
 \textbf{Divorce}                                                                                  &      -0.2960  &        0.276     &    -1.072  &         0.284        &       -0.837    &        0.245     \\
 \textbf{HH Composition Changed}                                                                          &      -0.0285  &        0.056     &    -0.505  &         0.614        &       -0.139    &        0.082     \\
 \textbf{Displacement}                                                                          &      -0.0222  &        0.032     &    -0.693  &         0.488        &       -0.085    &        0.041     \\
 \textbf{Legally Handicapped}                                                                &      -0.1187  &        0.093     &    -1.280  &         0.200        &       -0.301    &        0.063     \\
 \textbf{Unplanned Pregnancy}                                                                      &      -0.0525  &        0.065     &    -0.808  &         0.419        &       -0.180    &        0.075     \\
 \textbf{Separation}                                                                                 &       0.1072  &        0.177     &     0.607  &         0.544        &       -0.239    &        0.453     \\
 \textbf{Divorce * Single}                                                     &       0.3002  &        0.326     &     0.920  &         0.358        &       -0.340    &        0.940     \\
 \textbf{Displacement * Not Employed}                                    &      -0.0142  &        0.031     &    -0.454  &         0.650        &       -0.076    &        0.047     \\
 \textbf{Displacement * Other Empl. Status}                                           &      -0.0258  &        0.058     &    -0.444  &         0.657        &       -0.140    &        0.088     \\
 \textbf{Separation * Single}                                                    &      -0.0988  &        0.212     &    -0.465  &         0.642        &       -0.515    &        0.318     \\
 \textbf{Child Disorders * Time Since Occurrence}                                          &      -0.0060  &        0.004     &    -1.489  &         0.136        &       -0.014    &        0.002     \\
 \textbf{Child Death * Time Since Occurrence}                                                &      -0.0010  &        0.006     &    -0.149  &         0.882        &       -0.013    &        0.012     \\
 \textbf{Father Death * Time Since Occurrence}                                              &      -0.0025  &        0.001     &    -1.705  &         0.088        &       -0.005    &        0.000     \\
 \textbf{Household Person Death * Time Since Occurrence}                                      &       0.0067  &        0.005     &     1.236  &         0.217        &       -0.004    &        0.017     \\
 \textbf{Mother Death * Time Since Occurrence}                                              &       0.0021  &        0.002     &     1.342  &         0.179        &       -0.001    &        0.005     \\
 \textbf{Partner Death * Time Since Occurrence}                                            &       0.0043  &        0.003     &     1.586  &         0.113        &       -0.001    &        0.010     \\
 \textbf{Divorce * Time Since Occurrence}                                                        &       0.0126  &        0.011     &     1.154  &         0.248        &       -0.009    &        0.034     \\
 \textbf{Divorce * Single * Time Since Occurrence}                           &      -0.0087  &        0.011     &    -0.775  &         0.438        &       -0.031    &        0.013     \\
 \textbf{HH Composition Changed * Time Since Occurrence}                                        &      -0.0011  &        0.003     &    -0.381  &         0.703        &       -0.007    &        0.005     \\
 \textcolor{red}{\textbf{Displacement * Time Since Occurrence}}                                        &       \textcolor{red}{0.0024}  &        \textcolor{red}{0.001}     &     \textcolor{red}{2.547}  &         \textcolor{red}{0.011}        &        \textcolor{red}{0.001}    &        \textcolor{red}{0.004}     \\
 \textbf{Displacement * Time Since Occurrence * Not Employed}  &      -0.0018  &        0.001     &    -1.270  &         0.204        &       -0.005    &        0.001     \\
 \textbf{Displacement * Time Since Occurrence * Other Empl. Status}         &      -0.0005  &        0.003     &    -0.190  &         0.849        &       -0.006    &        0.005     \\
 \textbf{Handicapped * Time Since Occurrence}                    &       0.0025  &        0.005     &     0.528  &         0.597        &       -0.007    &        0.012     \\
 \textbf{Unplanned Pregnancy * Time Since Occurrence}                                &       0.0014  &        0.003     &     0.440  &         0.660        &       -0.005    &        0.008     \\
 \textbf{Separation * Time Since Occurrence}                                                      &      -0.0029  &        0.008     &    -0.375  &         0.708        &       -0.018    &        0.012     \\
 \textbf{Separation * Time Since Occurrence * Single}                         &       0.0036  &        0.008     &     0.456  &         0.649        &       -0.012    &        0.019     \\
 \end{tabular}
 %\caption{OLS Regression Results}
 \end{center}
\end{table}
\end{frame}

\begin{frame}[plain]
\fontsize{5}{6}\selectfont
\begin{table}
	\input{../../out/tables/reg_table_2.tex}
\end{table}
\end{frame}

\begin{frame}[t]\frametitle{Results}
\fontsize{6}{6}\selectfont
\begin{table}
\input{../../out/tables/reg_table_3.tex}
\end{table}
\fontsize{11}{9}\selectfont
\begin{itemize}
	\item<+-> no significant effect of the overall number of events
	\item<+-> none of the events changes the locus of control significantly
\end{itemize}
\end{frame}

% section results


\section{Extensions} % (fold)
\label{sec:extensions}

\subsection{Internal vs. External Types} % (fold)
\label{sub:internal_vs_external_types}
\begin{frame}[t]\frametitle{Extensions}
\textbf{Internal vs. External Types}
\begin{itemize}
	\item<+-> \citet{buddelmeyer2016}: insurance effect of internal locus of control against emotional pain
	\item<+-> compare effects of traumata on locus of control between internals and externals
	\item<+-> steps:
	\begin{enumerate}
		\item split sample by median locus of control, categorize individuals as externals and internals
		\item run previous regression with dummy for locus of control type and interaction terms between type and events
		\end{enumerate}
	\end{itemize}
\end{frame}

\subsection{Time Effects} % (fold)
\label{sub:time_effects}
\begin{frame}[t]\frametitle{Extensions}
\textbf{Time Effects}
\begin{itemize}
	\item<+-> two possible channels of effect of time since occurrence
	\item<+-> steps to test for diminished or increased effects over time:
	\begin{enumerate}
		\item restrict sample
		\item split continous time variable into dummies such that each dummy covers a twelve months interval
		\item run regression by interacting all events with time dummies
	\end{enumerate}
\end{itemize}
\end{frame}

\subsection{Age Effects and Positive Events} % (fold)
\label{sub:age_effects_positive_events}
\begin{frame}[t]\frametitle{Extensions}
\textbf{Age Effects and Positive Events}
\begin{itemize}
	\item<+-> age periods in which locus of control is more stable than in other periods
	\begin{itemize}
		\item compare effects of traumatic events between different age groups
	\end{itemize}
	\item<+-> investigate effect of positive events
	\end{itemize}
\end{frame}

\subsection{Common Trend Assumption} % (fold)
\label{sub:common_trend_assumption}

\begin{frame}[t]\frametitle{Extensions}

\textbf{Common Trend Assumption}
\begin{itemize}[<+->]
    \item In absence of treatment, control and treatment group would follow the same trend
    \item \citet{autor2003}: Include leads and lags dummy variables
    \item If leads are significant, common trend assumption has to be rejected
\end{itemize}

\end{frame}
% subsection common_trend_assumption (end)

% section extensions (end)

\begin{frame}[t]\frametitle{References}

\printbibliography

\end{frame}

\begin{frame}[plain, c]

\centering
Thank you for your attention!

\end{frame}

\bgroup
\setbeamercolor{background canvas}{bg=black}
\begin{frame}[plain]{}
\end{frame}
\egroup

\end{document}
