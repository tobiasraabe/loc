\documentclass{scrartcl}
\def\datum{\today}
\author{Tobias Raabe and Imke Strampe}
\title{Malleability of Locus of Control by Traumatic Events\\[0.5cm]
\Large Pre-Analysis Plan}

\usepackage[english]{babel}  % set language to adjust headings, automatic hypenation
\usepackage[utf8]{inputenc}  % allows to use umlauts and certain symbols
\usepackage[T1]{fontenc}  % typesetting, necessary for umlauts and accentuations
\usepackage{lmodern}  % Latin Modern Fonts are successors of standard fonts
\usepackage{amsmath, amssymb, amsthm}
\usepackage{graphicx}
\usepackage[space]{grffile}
\usepackage{hyperref}
\usepackage{xcolor}
\hypersetup{
    colorlinks,
    linkcolor={red!50!black},
    citecolor={blue!50!black},
    urlcolor={blue!80!black}
}

\usepackage[backend=biber, natbib=true, bibencoding=utf8, safeinputenc=true,
bibstyle=authoryear, citestyle=authoryear-comp, maxcitenames=2, mincitenames=1,
uniquelist=false, useprefix=true, maxbibnames=99, minbibnames=99, backref=true,
backrefstyle=three, doi=true, isbn=false, sortcites=false, dashed=false,
firstinits=true]{biblatex}

\addbibresource{references.bib}

\begin{document}
\maketitle

\section{Introduction} % (fold)
\label{sec:introduction}

Empirical studies investigating the effect of non-cognitive skills often rely on the
assumption that these skills are stable across the relevant time frame. In this study we
want to challenge this assumption by focusing on the locus of control and how it is
affected by traumatic events. This study is in line with the early work of
\citet{alesina2002} who looked at changes in trust due to traumatic events. An extension
of the framework to locus of control was done by \citet{cobb2013} who found that locus
of control is varying across time but seems to be stable for the working-age population.
\citet{buddelmeyer2016} built on top of their research and tried to find insurance
effects of an internal locus of control to traumatic events.

To limit researcher's degrees of freedom, this document serves as a pre- analysis plan
for our research project.\footnote{The plan inspired by
\href{https://blogs.worldbank.org/impactevaluations/a-pre-analysis-plan- checklist}{a
blog post by David McKenzie (2012)} and
\href{http://cega.berkeley.edu/assets/cega_events/92/Pre-Analysis_Plan_Template_Alejandro_Ganimian.pdf}{a
template written by Alejandro Ganimian}.}

% section introduction (end)

\section{Data} % (fold)
\label{sec:data}

Our data comes from the GSOEP and focuses on all participants from 2005-2015 which is
about 16,000 households or 30,000 people. Our main sample restriction is the elicitation
of the locus of control survey questions which are asked every five years from 2005 on.
Additionally, the questions are posed to every teenager turning 17, so we are able to
extend our sample with roughly 200 teenagers every year since 2006.

For the locus of control and other variables which are needed as controls, we need
access to the individual and deceased person questionnaire, mother and child (0-3 years)
and the youth questionnaire.

We do not think that our sample suffers from sample bias due to attrition since the DIW
Berlin, the institution in charge of the survey, makes tremendous effort to ensure that
people stay in the sample and that the sample is refreshed from time to time. The locus
of control questions have response rates of 99\% (19,800/20,000), 70\% (18,000/26,000)
and 98,5\% (26,600/27,000) for the years 2005, 2010, 2015, respectively. The lower
response rate in 2010 was due to a refreshment sample which did not receive the locus of
control items.

% section data (end)

\section{Empirical Analysis} % (fold)
\label{sec:empirical_analysis}

\subsection{Specifications} % (fold)
\label{sub:specifications}

For investigating how traumas can influence the locus of control, we would conduct three
specifications, a more general, a more specific one and one including just the number of
traumas experienced. In all specifications we would use change in locus of control as a
dependent variable. In the GSOEP, locus of control is measured by asking 10 questions
and we would use a principal component analysis to reduce this number of items.

Regarding the controls, we will follow the specifications by \citet{cobb2013}: dummies
for age groups, marital status, immigrant status, employment status, educational
qualification dummies and log household income. Furthermore, we include a gender dummy
and dummies for years into the regression.

We would also like to control for the time passed since the last shock to identify
depreciation effects of traumatic events, but we are not sure how to achieve this goal.
One idea is to include lagged variables of traumatic events.

In the first more general specification of our analysis, we would include traumatic
experiences concerning health, the social and the economic environment. We hypothesize
that all of these experiences shift the locus of control towards a more external locus
of control. Experiences concerning health are the birth of a child who is handicapped
and the change in the degree of disability of a household member. Social events which
could possibly influence the locus of control are a divorce, an unplanned pregnancy, a
relationship breakup, the death of a partner, father, mother, child or a household
member or any other family composition change. An experience concerning the economic
environment could be the loss of a job which we would include as a dummy in our first
specification.

In our second, more precise specification, we would drop some of the variables from the
three categories health, social and economic environment used before and include some in
another way. Again, we hypothesize that they all lead to a more external locus of
control. Concerning health, we would also include the birth of a handicapped child, but
the change in the degree of disability of a household member would only be included in
case of a change from 0 to 100\%. The traumas concerning the social environment would
also be an unplanned pregnancy, but neither a divorce, nor a relationship break-up as we
suppose that the latter incidents are in most cases not exogenous and would therefore
not necessarily lead to a more external locus of control. Moreover, regarding deaths, we
would drop deaths where the person investigated is not strongly affected by the loss. In
terms of the economic environment, we would include only the job losses which are rather
exogenous e.g. when the place of work closed or the person was dismissed by his or her
employer, but not if there was e.g. a mutual agreement with the employer or the job was
ended due to retirement.

In order to find out whether the relationship between change in locus of control and
negative life events varies with the number of negative life events reported, we include
the total number of traumas in our third specification instead of using all the dummies
for the different traumatic experiences. On top of that, we also include the total
number squared to test for any non- linearities. In this specification, we do not
control for the time between the occurrences of the traumas and the locus of control
measurement.

% subsection specifications (end)

\subsection{Additional Considerations} % (fold)
\label{sub:additional_considerations}


We need to make adjustments to the standard errors in our regression:
\begin{enumerate}
    \item We use robust standard errors to account for heteroscedasticity in the error
    terms as good practice.
    \item Since we have a panel, we need to use clustered standard errors on individual
    level to account for correlation between the same individual over time.
\end{enumerate}

As for now, we do not want to engage in data imputation and will drop cases in which
data is missing. Since we have no experience with the GSOEP, we are not sure whether
this is the final word.

% subsection additional_considerations (end)

% section empirical_analysis (end)
\printbibliography

\end{document}
