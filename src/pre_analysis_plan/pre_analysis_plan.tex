\documentclass{scrreprt}
\def\datum{\today}
\author{Tobias Raabe and Imke Strampe}
\title{Malleability of Locus of Control by Traumatic Events\\[0.5cm]
\Large Pre-Analysis Plan}

\usepackage[english]{babel}  % set language to adjust headings, automatic hypenation
\usepackage[utf8]{inputenc}  % allows to use umlauts and certain symbols
\usepackage[T1]{fontenc}  % typesetting, necessary for umlauts and accentuations
\usepackage{lmodern}  % Latin Modern Fonts are successors of standard fonts
\usepackage{amsmath, amssymb, amsthm}
\usepackage{graphicx}
\usepackage[space]{grffile}
\usepackage{hyperref}
\usepackage{xcolor}
\hypersetup{
    colorlinks,
    linkcolor={red!50!black},
    citecolor={blue!50!black},
    urlcolor={blue!80!black}
}

\begin{document}
\maketitle

\textbf{Sources:} Here are three templates on how to write a pre-analysis plan.
We should follow their examples.
\begin{enumerate}
    \item \url{https://blogs.worldbank.org/impactevaluations/a-pre-analysis-plan-checklist}
    \item \url{https://www.povertyactionlab.org/sites/default/files/documents/Pre-Analysis\%20Plan_Wydick_2-12-13.pdf}
    \item \url{http://cega.berkeley.edu/assets/cega_events/92/Pre-Analysis_Plan_Template_Alejandro_Ganimian.pdf}
\end{enumerate}

\textbf{Todo}
\begin{enumerate}
    \item ...
\end{enumerate}

For investigating how traumas can influence the locus of control, we would
conduct two specifications, a more general one and a more specific one. In the
first more general specification of our analysis, we would include traumatic
experiences concerning health, the social and the economic environment which
are available in the GSOEP. We hypothesize that all of these experiences shift
the locus of control towards a more external locus of control. Experiences
concerning health are the birth of a child who is handicapped and the change in
the degree of disability of a household member. Social events which could
possibly influence the locus of control are a divorce, an unplanned pregnancy,
a relationship breakup, the death of a partner, father, mother, child or a
household member or any other family composition change. An experience
concerning the economic environment could be the loss of a job which we would
include as a dummy in our first specification. In our second, more precise
specification, we would drop some of the variables from the three categories
health, social and economic environment used before and include some in another
way. Again, we hypothesize that they all lead to a more external locus of
control. Concerning health, we would also include the birth of a handicapped
child, but the change in the degree of disability of a household member would
only be included in case of a change from 0 to 100\%.

The traumas concerning the social environment would also be an unplanned
pregnancy, but neither a divorce, nor a relationship break-up as we suppose
that the latter incidents are in most cases not exogenous and would therefore
not necessarily lead to a more external locus of control. Moreover, regarding
deaths, we would drop deaths where the person investigated is not strongly
affected by the loss. In terms of the economic environment, we would include
only the job losses which are rather exogenous e.g. when the place of work
closed or the person was dismissed by his or her employer, but not if there was
e.g. a mutual agreement with the employer or the job was ended due to
retirement.
\end{document}